\documentclass{article}
\usepackage{tikz,lipsum,lmodern}
\usepackage[most]{tcolorbox}
\usepackage[paperheight=10.75in,paperwidth=7.25in,margin=1in,heightrounded]{geometry}
\usepackage{graphicx}
\usepackage{blindtext}
\usepackage{ragged2e}
\usepackage[space]{grffile}
\usepackage[utf8]{inputenc}

\graphicspath{{"/home/silasUser/.local/share/Anki2/User 1/collection.media/"}}

\title{todo extract deck name}

\begin{document}
%*********************
\section{Generelles}
%---------------------
\begin{tcolorbox}[colback=white!10!white,colframe=lightgray!75!black,
  savelowerto=\jobname_ex.tex]

\begin{center}
 Erl\"autern Sie die Motivation hinter der Datenbanktheorie. Gehen Sie insbesondere auf die 
Probleme
 ein, und beschreiben Sie was es heißt wenn ein Datum 
integer
ist. 

\end{center}

\tcblower

\justifying
\includegraphics[width=.9\textwidth]{paste-3440268804097.jpg}
\end{tcolorbox}
%---------------------
\begin{tcolorbox}[colback=white!10!white,colframe=lightgray!75!black,
  savelowerto=\jobname_ex.tex]

\begin{center}
 Erläutern Sie die Motivation hinter der Datenbanktheorie. Gehen Sie insbesondere auf die 
Probleme
 ein, und beschreiben Sie was es heißt wenn ein Datum 
integer
ist. 

\end{center}

\tcblower

\justifying
\includegraphics[width=.9\textwidth]{paste-3440268804097.jpg}
\end{tcolorbox}

\end{document}


