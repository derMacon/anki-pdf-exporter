\documentclass{article}
\usepackage{tikz,lipsum,lmodern}
\usepackage[most]{tcolorbox}
\usepackage[paperheight=10.75in,paperwidth=7.25in,margin=1in,heightrounded]{geometry}
\usepackage{graphicx}
\usepackage{blindtext}
\usepackage{ragged2e}
\usepackage[space]{grffile}
\usepackage[utf8]{inputenc}
\usepackage[export]{adjustbox}

\usepackage{fancyhdr}
\pagestyle{fancy}
\fancyhf{}
\rhead{\rightmark}
\chead{\thepart}
\lhead{\nouppercase{\leftmark}}
\cfoot{\thepage}
\graphicspath{{"/home/silasUser/.local/share/Anki2/User 1/collection.media/"}}

\title{todo extract deck name}
\author{Silas Hoffmann, inf103088}
\date{\today}



\begin{document}
\maketitle
\clearpage
\tableofcontents
\clearpage

%*********************
\section{Generelles}
%---------------------
\begin{tcolorbox}[colback=white!10!white,colframe=lightgray!75!black,
  savelowerto=\jobname_ex.tex,breakable,enhanced,lines before break=40]

\begin{center}
 Erläutern Sie die Motivation hinter der Datenbanktheorie. Gehen Sie insbesondere auf die 
\textbf{Probleme
} ein, und beschreiben Sie was es heißt wenn ein Datum 
\textbf{integer
}ist. 

\end{center}

\tcblower

\justifying
\begin{figure}
\includegraphics[max width=.9\textwidth]{paste-3440268804097.jpg}
\end{figure}
\end{tcolorbox}
%---------------------
\begin{tcolorbox}[colback=white!10!white,colframe=lightgray!75!black,
  savelowerto=\jobname_ex.tex,breakable,enhanced,lines before break=40]

\begin{center}
 Erläutern Sie das 
\textbf{ANSI-SPARC
}Schema. Welches Ziel wird hierbei verfolgt. 

\end{center}

\tcblower

\justifying
\begin{figure}
\includegraphics[max width=.9\textwidth]{paste-4144643440641.jpg}
\end{figure}
\end{tcolorbox}
%---------------------
\begin{tcolorbox}[colback=white!10!white,colframe=lightgray!75!black,
  savelowerto=\jobname_ex.tex,breakable,enhanced,lines before break=40]

\begin{center}
 Wie werden Daten auf einer Magnetplatte 
\textit{adressiert?
}Nennen Sie sowohl den Überbegriff als auch die einzelnen Komponenten aus denen sich dieser zusammensetzt. 

\end{center}

\tcblower

\justifying
\begin{figure}
\includegraphics[max width=.9\textwidth]{paste-11072425689089.jpg}
\end{figure}
\end{tcolorbox}
%---------------------
\begin{tcolorbox}[colback=white!10!white,colframe=lightgray!75!black,
  savelowerto=\jobname_ex.tex,breakable,enhanced,lines before break=40]

\begin{center}
 Wägen Sie ab warum es sinnvoll sein kann bei der Speicherung von Daten auf einer Magnetplatte auf größere Datenblöcke zu setzen. 

\end{center}

\tcblower

\justifying
Bei zu kleinen Datenblöcken müssen die zu speichernden Daten öfter aufgeteilt werden. Da diese im Zweifelsfall über die gesamte Platte verteilt sein können wird viel Zeit für diverse Schreib- und Lesevorgänge benötigt.
Bei zu großen Datenblöcken kann es vorkommen, dass eine kleines Datum (z.B. eine Zahl) nur einen geringen Teil der verfügbaren Blockgröße verwendet.

\end{tcolorbox}
%---------------------
\begin{tcolorbox}[colback=white!10!white,colframe=lightgray!75!black,
  savelowerto=\jobname_ex.tex,breakable,enhanced,lines before break=40]

\begin{center}
 Erläutern Sie grob wie ein Raid-System funktioniert. Was versteht man unter dem sog. 
\textbf{Striping
}sowie dem 
\textbf{Mirroring
}? 

\end{center}

\tcblower

\justifying
\begin{figure}
\includegraphics[max width=.9\textwidth]{paste-12695923326977.jpg}
\end{figure}
\end{tcolorbox}
%---------------------
\begin{tcolorbox}[colback=white!10!white,colframe=lightgray!75!black,
  savelowerto=\jobname_ex.tex,breakable,enhanced,lines before break=40]

\begin{center}
 Erläutern Sie was man unter einem 
\textbf{Raid-4 System
}versteht. Wie wird hierbei die Parität ermittelt? 

\end{center}

\tcblower

\justifying
\begin{figure}
\includegraphics[max width=.9\textwidth]{paste-13030930776065.jpg}
\end{figure}\begin{figure}
\includegraphics[max width=.9\textwidth]{paste-13597866459137.jpg}
\end{figure}
\end{tcolorbox}
%---------------------
\begin{tcolorbox}[colback=white!10!white,colframe=lightgray!75!black,
  savelowerto=\jobname_ex.tex,breakable,enhanced,lines before break=40]

\begin{center}
 
\begin{figure}
\includegraphics[max width=.9\textwidth]{paste-18150531792897.jpg}
\end{figure}\begin{figure}
\includegraphics[max width=.9\textwidth]{paste-18163416694785.jpg}
\end{figure} 

\end{center}

\tcblower

\justifying
\begin{figure}
\includegraphics[max width=.9\textwidth]{paste-18176301596673.jpg}
\end{figure}
\end{tcolorbox}
%---------------------
\begin{tcolorbox}[colback=white!10!white,colframe=lightgray!75!black,
  savelowerto=\jobname_ex.tex,breakable,enhanced,lines before break=40]

\begin{center}
 Warum erhöht ein Raid 4 die Ausfallsicherheit, ersetzt aber kein Backup? 

\end{center}

\tcblower

\justifying
Manuelles Löschen führt weiterhin zum Datenverlust.

\end{tcolorbox}
%---------------------
\begin{tcolorbox}[colback=white!10!white,colframe=lightgray!75!black,
  savelowerto=\jobname_ex.tex,breakable,enhanced,lines before break=40]

\begin{center}
 Weswegen wird in der Datenbanktheorie eine möglichst effiziente Pufferverwaltung angestrebt? Beschreiben Sie grob wie so eine aussehen könnte. 

\end{center}

\tcblower

\justifying
\begin{figure}
\includegraphics[max width=.9\textwidth]{paste-7584912244737.jpg}
\end{figure}
\end{tcolorbox}
%---------------------
\begin{tcolorbox}[colback=white!10!white,colframe=lightgray!75!black,
  savelowerto=\jobname_ex.tex,breakable,enhanced,lines before break=40]

\begin{center}
 Was versteht man unter dem 
\textbf{Mapping
}von Speicheradressen? 

\end{center}

\tcblower

\justifying
Zuordnung von Speicheradressen auf Scheiben, Spuren und Sektoren.
\begin{figure}
\includegraphics[max width=.9\textwidth]{paste-7791070674945.jpg}
\end{figure}
\end{tcolorbox}
%---------------------
\begin{tcolorbox}[colback=white!10!white,colframe=lightgray!75!black,
  savelowerto=\jobname_ex.tex,breakable,enhanced,lines before break=40]

\begin{center}
 Beschreiben Sie generell den Begriff der Speicherverwaltung. 

\end{center}

\tcblower

\justifying
\begin{figure}
\includegraphics[max width=.9\textwidth]{paste-7786775707649.jpg}
\end{figure}
\end{tcolorbox}
%---------------------
\begin{tcolorbox}[colback=white!10!white,colframe=lightgray!75!black,
  savelowerto=\jobname_ex.tex,breakable,enhanced,lines before break=40]

\begin{center}
 Beschreiben Sie grob den Begriff der Speicherzuordnung. Mit welchem Hilfsmittel ist es hierbei möglich die Datenblöcke zu lokalisieren? Müssen die Datenblöcke stets die gleich Länge aufweisen? 

\end{center}

\tcblower

\justifying
\begin{figure}
\includegraphics[max width=.9\textwidth]{paste-8504035246081.jpg}
\end{figure}
\end{tcolorbox}
%---------------------
\begin{tcolorbox}[colback=white!10!white,colframe=lightgray!75!black,
  savelowerto=\jobname_ex.tex,breakable,enhanced,lines before break=40]

\begin{center}
 Was sind die Ziele der 
\textbf{Pufferverwaltung
}? 

\end{center}

\tcblower

\justifying
\begin{figure}
\includegraphics[max width=.9\textwidth]{paste-8598524526593.jpg}
\end{figure}
\end{tcolorbox}
%---------------------
\begin{tcolorbox}[colback=white!10!white,colframe=lightgray!75!black,
  savelowerto=\jobname_ex.tex,breakable,enhanced,lines before break=40]

\begin{center}
 Wo werden die Pufferdaten zwischengelagert? 

\end{center}

\tcblower

\justifying
In sog. 
\textbf{Seiten
}im Hauptspeicher (RAM).
\begin{figure}
\includegraphics[max width=.9\textwidth]{paste-8594229559297.jpg}
\end{figure}
\end{tcolorbox}
%---------------------
\begin{tcolorbox}[colback=white!10!white,colframe=lightgray!75!black,
  savelowerto=\jobname_ex.tex,breakable,enhanced,lines before break=40]

\begin{center}
 Beschreiben Sie anhand der Datenbanken vom Hersteller 
\textbf{Oracle
}was man unter einem 
\textit{Segment 
}sowie einem 
\textit{Extent 
}versteht. Wie kann es sein, dass die Größe des 
\textit{Extents 
}von den kleinsten logischen Einheiten einer Festplatte / SSD abweichen kann? 

\end{center}

\tcblower

\justifying
\begin{figure}
\includegraphics[max width=.9\textwidth]{paste-9783935500289.jpg}
\end{figure}Da eine Datenbank i.d.R. ein eigenes Filesystem anlegt ist es komplett losgelöst von sämtlichen Restriktionen von Festspeicher / Betriebssystem.

\end{tcolorbox}
%---------------------
\begin{tcolorbox}[colback=white!10!white,colframe=lightgray!75!black,
  savelowerto=\jobname_ex.tex,breakable,enhanced,lines before break=40]

\begin{center}
 Was ermöglicht das Konzept eines 
\textbf{Segments
}in der Datenbanktheorie? 

\end{center}

\tcblower

\justifying
\begin{figure}
\includegraphics[max width=.9\textwidth]{paste-10269266804737.jpg}
\end{figure}
\end{tcolorbox}
%---------------------
\begin{tcolorbox}[colback=white!10!white,colframe=lightgray!75!black,
  savelowerto=\jobname_ex.tex,breakable,enhanced,lines before break=40]

\begin{center}
 Beschreiben Sie das Prinzip der 
\textbf{direkten
}Seitenadressierung an einem Bild. 

\end{center}

\tcblower

\justifying
\begin{figure}
\includegraphics[max width=.9\textwidth]{paste-10690173599745.jpg}
\end{figure}
\end{tcolorbox}
%---------------------
\begin{tcolorbox}[colback=white!10!white,colframe=lightgray!75!black,
  savelowerto=\jobname_ex.tex,breakable,enhanced,lines before break=40]

\begin{center}
 Beschreiben Sie das Prinzip der 
\textbf{indirekten
}Speicheradressierung. Gehen Sie insbesondere auf die beiden verwendeten Hilfsstrukturen ein. 

\end{center}

\tcblower

\justifying
\begin{figure}
\includegraphics[max width=.9\textwidth]{paste-11016591114241.jpg}
\end{figure}
\end{tcolorbox}
%---------------------
\begin{tcolorbox}[colback=white!10!white,colframe=lightgray!75!black,
  savelowerto=\jobname_ex.tex,breakable,enhanced,lines before break=40]

\begin{center}
 Beschreiben Sie das Prinzip der 
\textbf{direkten
}\textit{Einbringunsstrategie.
}Welche Risiken können hierbei auftreten und wie kann man diesen entgegenwirken? 

\end{center}

\tcblower

\justifying
\begin{figure}
\includegraphics[max width=.9\textwidth]{paste-11768210391041.jpg}
\end{figure}WAL-Prinzip: Man schreibt einen Log über die Dinge die als nächstes getan werden sollen (wie eine Todo-Liste). Falls hierbei ein Stromausfall dazwischen kommt weiß man genau welche Daten noch nicht kopiert wurden / noch gelöscht werden müssen um die Daten konsistent zu halten.

\end{tcolorbox}
%---------------------
\begin{tcolorbox}[colback=white!10!white,colframe=lightgray!75!black,
  savelowerto=\jobname_ex.tex,breakable,enhanced,lines before break=40]

\begin{center}
 Beschreiben Sie den Begriff der
\textbf{verzögerten Einbringungsstrategie
}am Beispiel des 
\textit{Schattenspeicherkonzepts.
} 

\end{center}

\tcblower

\justifying
\begin{figure}
\includegraphics[max width=.9\textwidth]{paste-12545599471617.jpg}
\end{figure}\begin{figure}
\includegraphics[max width=.9\textwidth]{paste-12725988098049.jpg}
\end{figure}\begin{figure}
\includegraphics[max width=.9\textwidth]{paste-12738872999937.jpg}
\end{figure}
\end{tcolorbox}
%---------------------
\begin{tcolorbox}[colback=white!10!white,colframe=lightgray!75!black,
  savelowerto=\jobname_ex.tex,breakable,enhanced,lines before break=40]

\begin{center}
 Erläutern Sie das Konzept des 
\textbf{Schattenspeicherkonzepts.
} 

\end{center}

\tcblower

\justifying
\begin{figure}
\includegraphics[max width=.9\textwidth]{paste-6648609374209.jpg}
\end{figure}\begin{figure}
\includegraphics[max width=.9\textwidth]{paste-6661494276097.jpg}
\end{figure}
\end{tcolorbox}
%---------------------
\begin{tcolorbox}[colback=white!10!white,colframe=lightgray!75!black,
  savelowerto=\jobname_ex.tex,breakable,enhanced,lines before break=40]

\begin{center}
 Erläutern Sie wie die Verwaltung des Puffers funktioniert? Gehen Sie insbesondere auf die Funktion des Pufferrahmens sowie des Pufferkontrollblocks ein. 

\end{center}

\tcblower

\justifying
\begin{figure}
\includegraphics[max width=.9\textwidth]{paste-8693013807105.jpg}
\end{figure}
\end{tcolorbox}
%---------------------
\begin{tcolorbox}[colback=white!10!white,colframe=lightgray!75!black,
  savelowerto=\jobname_ex.tex,breakable,enhanced,lines before break=40]

\begin{center}
 Welche Kontroll- und Zuteilungsaufgaben besitzt die Speicherverwaltung einer Datenbank? 

\end{center}

\tcblower

\justifying
\begin{figure}
\includegraphics[max width=.9\textwidth]{paste-9745280794625.jpg}
\end{figure}
\end{tcolorbox}
%---------------------
\begin{tcolorbox}[colback=white!10!white,colframe=lightgray!75!black,
  savelowerto=\jobname_ex.tex,breakable,enhanced,lines before break=40]

\begin{center}
 Gehen Sie auf die verschieden Szenarien ein welche beim Arbeiten mit einem Puffer auftreten können. 

\end{center}

\tcblower

\justifying
\begin{figure}
\includegraphics[max width=.9\textwidth]{paste-10861972291585.jpg}
\end{figure}
\end{tcolorbox}
%---------------------
\begin{tcolorbox}[colback=white!10!white,colframe=lightgray!75!black,
  savelowerto=\jobname_ex.tex,breakable,enhanced,lines before break=40]

\begin{center}
 Geben Sie eine grobe Übersicht über die verschiedenen Suchstrategien der Pufferverwaltung. 

\end{center}

\tcblower

\justifying
\begin{figure}
\includegraphics[max width=.9\textwidth]{paste-11781095292929.jpg}
\end{figure}
\end{tcolorbox}
%---------------------
\begin{tcolorbox}[colback=white!10!white,colframe=lightgray!75!black,
  savelowerto=\jobname_ex.tex,breakable,enhanced,lines before break=40]

\begin{center}
 Erläutern Sie nun noch kurz, wozu bei einem Router (oder einer Arbeitsstation) eine Wegwahl- bzw. Routingtabelle eigentlich dient. Was ist das bzw. was wird hier prinzipielle gespeichert? 

\end{center}

\tcblower

\justifying
In paketvermittelten Netzen werden Ende-zu-Ende-Verbindungen zwischen zwei Endgeräten dadurch hergestellt, dass Datenpakete von einer sendenden Station zu einer Empfangsstation übertragen werden. Die Stationen müssen über ihre Netzwerkadressen eindeutig identifizierbar sein. Zwischen den beiden Stationen liegen Router, die die Datenpakete gemäß ihrer Routingtabellen von der Sendestation über den ersten Router zum zweiten, zum dritten usw. weiterleiten, bis der letzte Router auf dem Weg durch das Datenpaketnetz die Datenpakete an die Empfängerstation ausliefert.
Routingtabellen werden von den Routing-Protokollen für die Pfadermittlung erstellt und können je nach Routing-Protokoll unterschiedlich sein. Beim statischen Routing ist der Routingpfad fest vorgegeben. Beim dynamischen Routing werden die Routingtabellen durch aktuelle Routinginformationen aktualisiert. Die Aktualisierung übernehmen Routing-Protokolle, die die Informationen eintragen. Das können Information über die Verknüpfungen der Netzwerke, bzw. der Endgeräte und der Hops mit den dazu gehörenden IP-Adressen sein, oder die Routing-Metriken mit der die Route festgelegt wurde.
\begin{figure}
\includegraphics[max width=.9\textwidth]{paste-3264175144961.png}
\end{figure}
\end{tcolorbox}
%---------------------
\begin{tcolorbox}[colback=white!10!white,colframe=lightgray!75!black,
  savelowerto=\jobname_ex.tex,breakable,enhanced,lines before break=40]

\begin{center}
 Was versteht man unter einem 
\textbf{logischen
}Seitenreferenzstring? 

\end{center}

\tcblower

\justifying
\begin{figure}
\includegraphics[max width=.9\textwidth]{paste-4050154160129.jpg}
\end{figure}\begin{figure}
\includegraphics[max width=.9\textwidth]{paste-4080218931201.jpg}
\end{figure}
\end{tcolorbox}
%---------------------
\begin{tcolorbox}[colback=white!10!white,colframe=lightgray!75!black,
  savelowerto=\jobname_ex.tex,breakable,enhanced,lines before break=40]

\begin{center}
 Geben Sie eine kurze Aufschlüsselung über die möglichen Suchstrategien bei der 
\textit{Pufferverwaltung.
} 

\end{center}

\tcblower

\justifying

\end{tcolorbox}
%---------------------
\begin{tcolorbox}[colback=white!10!white,colframe=lightgray!75!black,
  savelowerto=\jobname_ex.tex,breakable,enhanced,lines before break=40]

\begin{center}
 Erläutern Sie die 
\textbf{Vorgehensweise 
}bei der direkten Suche im Pufferrahmen, wie viele Rahmen werden durchschnittlich durchsucht und was ist der wesentliche Nachteil dieser Technik? 

\end{center}

\tcblower

\justifying
\begin{figure}
\includegraphics[max width=.9\textwidth]{paste-4780298600449.jpg}
\end{figure}
\end{tcolorbox}
%---------------------
\begin{tcolorbox}[colback=white!10!white,colframe=lightgray!75!black,
  savelowerto=\jobname_ex.tex,breakable,enhanced,lines before break=40]

\begin{center}
 Erläutern Sie wie eine indirekten Suche mittels einer Hashfunktion funktioniert. Gehen Sie hierbei insbesondere auf eine sog. Überlaufkette ein. 

\end{center}

\tcblower

\justifying
\begin{figure}
\includegraphics[max width=.9\textwidth]{paste-3809635991553.png}
\end{figure}\begin{figure}
\includegraphics[max width=.9\textwidth]{paste-3831110828033.png}
\end{figure}
\end{tcolorbox}
%---------------------
\begin{tcolorbox}[colback=white!10!white,colframe=lightgray!75!black,
  savelowerto=\jobname_ex.tex,breakable,enhanced,lines before break=40]

\begin{center}
 Wie funktioniert die Suche nach den im Puffer über eine 
\textit{direkt adressierte 
}\textbf{Umsetzungstabelle
}? Welchen wesentlichen Nachteil bringt dieses Verfahren mit sich? 

\end{center}

\tcblower

\justifying
\begin{figure}
\includegraphics[max width=.9\textwidth]{paste-4913442586625 (1).png}
\end{figure}
\end{tcolorbox}
%---------------------
\begin{tcolorbox}[colback=white!10!white,colframe=lightgray!75!black,
  savelowerto=\jobname_ex.tex,breakable,enhanced,lines before break=40]

\begin{center}
 Wie funktioniert die indirekte Suche nach einer Seite im Pufferrahmen mittels einer 
\textit{unsortierten
} Liste?Gehen Sie insbesondere auf die Vor-/Nachteile sowie die Laufzeit des Verfahrens ein. 

\end{center}

\tcblower

\justifying
\begin{figure}
\includegraphics[max width=.9\textwidth]{paste-5862630359041.png}
\end{figure}
\end{tcolorbox}
%---------------------
\begin{tcolorbox}[colback=white!10!white,colframe=lightgray!75!black,
  savelowerto=\jobname_ex.tex,breakable,enhanced,lines before break=40]

\begin{center}
 Wie funktioniert die indirekte Suche nach einer Seite im Pufferrahmen mittels einer
\textit{sortierten
} Liste? Gehen Sie insbesondere auf die Vor-/Nachteile sowie die Laufzeit des Verfahrens ein. 

\end{center}

\tcblower

\justifying
\begin{figure}
\includegraphics[max width=.9\textwidth]{paste-5935644803073.png}
\end{figure}
\end{tcolorbox}
%---------------------
\begin{tcolorbox}[colback=white!10!white,colframe=lightgray!75!black,
  savelowerto=\jobname_ex.tex,breakable,enhanced,lines before break=40]

\begin{center}
 Wie funktioniert die indirekte Suche nach einer Seite im Pufferrahmen mittels einer
\textit{sortierten
}\textit{mit verketteten Einträgen
}Liste? Gehen Sie insbesondere auf die Vorteile des Verfahrens ein. 

\end{center}

\tcblower

\justifying
\begin{figure}
\includegraphics[max width=.9\textwidth]{paste-7962869366785 (1).png}
\end{figure}Warum sollte man Zeiger verwenden?
:
Wenn ein neues Element eigefügt / gelöscht wird muss man nicht die komplette Restliste umkopieren um Platz für eine neues Element zu machen / löschen  Man setzt einfach 2 Zeiger neu.

\end{tcolorbox}
%---------------------
\begin{tcolorbox}[colback=white!10!white,colframe=lightgray!75!black,
  savelowerto=\jobname_ex.tex,breakable,enhanced,lines before break=40]

\begin{center}
 Welche wesentlichen Aufrufe einer Datenbank muss eine Pufferverwaltung bewerkstelligen können? 

\end{center}

\tcblower

\justifying
\begin{figure}
\includegraphics[max width=.9\textwidth]{paste-10320806412289.png}
\end{figure}
\end{tcolorbox}
%---------------------
\begin{tcolorbox}[colback=white!10!white,colframe=lightgray!75!black,
  savelowerto=\jobname_ex.tex,breakable,enhanced,lines before break=40]

\begin{center}
 Welches Problem wird mit der Speicherzuteilung im Puffer gelöst. Welche Strategie wird hierbei mit welchen Eigenschaften verwendet? 

\end{center}

\tcblower

\justifying
\begin{figure}
\includegraphics[max width=.9\textwidth]{paste-12592844111873.png}
\end{figure}
\end{tcolorbox}
%---------------------
\begin{tcolorbox}[colback=white!10!white,colframe=lightgray!75!black,
  savelowerto=\jobname_ex.tex,breakable,enhanced,lines before break=40]

\begin{center}
 Geben Sie eine grobe Klassifikation der 
\textbf{Speicherzuteilungsstrategien
}, zwischen welchen wesentlichen Strategien gilt es hierbei zu unterscheiden? 

\end{center}

\tcblower

\justifying
\begin{figure}
\includegraphics[max width=.9\textwidth]{paste-12970801233921.png}
\end{figure}
\end{tcolorbox}
%---------------------
\begin{tcolorbox}[colback=white!10!white,colframe=lightgray!75!black,
  savelowerto=\jobname_ex.tex,breakable,enhanced,lines before break=40]

\begin{center}
 Was versteht man unter einer 
\textit{statischen / dynamischen / seitenbezogenen
}Speicherzuteilungsstrategie? 

\end{center}

\tcblower

\justifying
\begin{figure}
\includegraphics[max width=.9\textwidth]{paste-13872744366081.png}
\end{figure}
\end{tcolorbox}
%---------------------
\begin{tcolorbox}[colback=white!10!white,colframe=lightgray!75!black,
  savelowerto=\jobname_ex.tex,breakable,enhanced,lines before break=40]

\begin{center}
 Grenzen Sie das 
\textbf{Demand-Paging-
}vom 
\textbf{Prepaging-Verfahren
}ab. 

\end{center}

\tcblower

\justifying
\begin{figure}
\includegraphics[max width=.9\textwidth]{paste-14345190768641.png}
\end{figure}\begin{figure}
\includegraphics[max width=.9\textwidth]{paste-14358075670529.png}
\end{figure}
\end{tcolorbox}
%---------------------
\begin{tcolorbox}[colback=white!10!white,colframe=lightgray!75!black,
  savelowerto=\jobname_ex.tex,breakable,enhanced,lines before break=40]

\begin{center}
 Erläutern Sie das 
\textbf{FIFO
}Ersetzungsverfahren, für welchen Anwendungsfall eigent sich dieses Verfahren am besten? 

\end{center}

\tcblower

\justifying
\begin{figure}
\includegraphics[max width=.9\textwidth]{paste-15019500634113.png}
\end{figure}
\end{tcolorbox}
%---------------------
\begin{tcolorbox}[colback=white!10!white,colframe=lightgray!75!black,
  savelowerto=\jobname_ex.tex,breakable,enhanced,lines before break=40]

\begin{center}
 Erläutern Sie das
\textbf{LFU
}Ersetzungsverfahren. Welchen wesentlichen Nachteil bringt dieses Verfahren mit sich und wie kann man dem entgegenwirken? 

\end{center}

\tcblower

\justifying
\begin{figure}
\includegraphics[max width=.9\textwidth]{paste-15079630176257.png}
\end{figure}
\end{tcolorbox}
%---------------------
\begin{tcolorbox}[colback=white!10!white,colframe=lightgray!75!black,
  savelowerto=\jobname_ex.tex,breakable,enhanced,lines before break=40]

\begin{center}
 Erläutern Sie das 
\textbf{LRU
}Verfahren. Welche Spezialisierungen / genauen Ausprägungen dieses Verfahrens gibt es? 

\end{center}

\tcblower

\justifying
\begin{figure}
\includegraphics[max width=.9\textwidth]{paste-18150531792897.png}
\end{figure}
\end{tcolorbox}
%---------------------
\begin{tcolorbox}[colback=white!10!white,colframe=lightgray!75!black,
  savelowerto=\jobname_ex.tex,breakable,enhanced,lines before break=40]

\begin{center}
 Wie lässt sich ein Dropdown Text mit HTML implementieren, beschreiben Sie die Syntax am folgenden Beispiel:
\begin{figure}
\includegraphics[max width=.9\textwidth]{paste-5252745003009.jpg}
\end{figure} 

\end{center}

\tcblower

\justifying
\begin{figure}
\includegraphics[max width=.9\textwidth]{paste-5265629904897.jpg}
\end{figure}
\end{tcolorbox}
%---------------------
\begin{tcolorbox}[colback=white!10!white,colframe=lightgray!75!black,
  savelowerto=\jobname_ex.tex,breakable,enhanced,lines before break=40]

\begin{center}
 Erläutern Sie die 
\textbf{CLOCK / Second Chance
}Ersetzungsstrategie. 

\end{center}

\tcblower

\justifying
\begin{figure}
\includegraphics[max width=.9\textwidth]{paste-7885559955457.png}
\end{figure}
\end{tcolorbox}
%---------------------
\begin{tcolorbox}[colback=white!10!white,colframe=lightgray!75!black,
  savelowerto=\jobname_ex.tex,breakable,enhanced,lines before break=40]

\begin{center}
 
\begin{figure}
\includegraphics[max width=.9\textwidth]{paste-8349416423425.png}
\end{figure} 

\end{center}

\tcblower

\justifying
\begin{figure}
\includegraphics[max width=.9\textwidth]{paste-8594229559297.png}
\end{figure}
\end{tcolorbox}
%---------------------
\begin{tcolorbox}[colback=white!10!white,colframe=lightgray!75!black,
  savelowerto=\jobname_ex.tex,breakable,enhanced,lines before break=40]

\begin{center}
 Was versteht man unter einem 
\textit{Page Fault?
} 

\end{center}

\tcblower

\justifying
Ein Seitenfehler (engl. page fault) tritt bei Betriebssystemen mit Virtueller Speicherverwaltung und Paging auf, wenn ein Programm auf einen Speicherbereich zugreift, der sich gerade nicht im Hauptspeicher befindet, sondern beispielsweise auf die Festplatte ausgelagert wurde oder wenn zu der betreffenden Adresse gerade kein Beschreibungseintrag in der MMU verfügbar ist. Als unmittelbare Folge des Seitenfehlers kommt es zu einer synchronen Programmunterbrechung (engl.: synchronous exception (fault)). Das Betriebssystem sorgt nun dafür, dass der angeforderte Speicherbereich wieder in den Hauptspeicher geladen wird oder der fehlende MMU-Eintrag nachgeladen wird, damit das Programm darauf zugreifen kann. Ein Seitenfehler ist daher kein Fehler im eigentlichen Sinne. Der Anwender spürt von diesem Vorgang nichts, maximal eine Verlangsamung des Programms, das den Seitenfehler verursachte, da das Laden der Seite oder das Bearbeiten des Vorgangs einen kurzen Augenblick benötigt. Andere Programme oder Prozesse sind davon nicht betroffen.
\begin{figure}
\includegraphics[max width=.9\textwidth]{paste-9814000271361.png}
\end{figure}
\end{tcolorbox}
%---------------------
\begin{tcolorbox}[colback=white!10!white,colframe=lightgray!75!black,
  savelowerto=\jobname_ex.tex,breakable,enhanced,lines before break=40]

\begin{center}
 Erläutern Sie den 
\textit{Database Fault
} 

\end{center}

\tcblower

\justifying
\begin{figure}
\includegraphics[max width=.9\textwidth]{paste-10900626997249.png}
\end{figure}
\end{tcolorbox}
%---------------------
\begin{tcolorbox}[colback=white!10!white,colframe=lightgray!75!black,
  savelowerto=\jobname_ex.tex,breakable,enhanced,lines before break=40]

\begin{center}
 Erläutern Sie was man unter einem 
\textit{Double page fault 
}versteht. 

\end{center}

\tcblower

\justifying
\begin{figure}
\includegraphics[max width=.9\textwidth]{paste-12043088297985.png}
\end{figure}
\end{tcolorbox}

\end{document}


