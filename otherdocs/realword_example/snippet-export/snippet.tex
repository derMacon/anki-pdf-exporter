\documentclass{article}
\usepackage{tikz,lipsum,lmodern}
\usepackage[ngerman]{babel}
\usepackage[most]{tcolorbox}
\usepackage[paperheight=10.75in,paperwidth=7.25in,margin=1in,heightrounded]{geometry}
\usepackage{graphicx}
\usepackage{blindtext}
\usepackage{ragged2e}
\usepackage[space]{grffile}
\usepackage[utf8]{inputenc}
\usepackage[export]{adjustbox}
\usepackage{hyperref}
\hypersetup{
    colorlinks,
    citecolor=black,
    filecolor=black,
    linkcolor=black,
    urlcolor=black
}
\usepackage{fancyhdr}
\pagestyle{fancy}
\fancyhf{}
\rhead{\rightmark}
\chead{\thepart}
\lhead{\nouppercase{\leftmark}}
\cfoot{\thepage}
\graphicspath{{"/home/silasUser/.local/share/Anki2/User 1/collection.media/"}}

\title{snippet}
\author{Silas Hoffmann, inf103088}
\date{\today}



\begin{document}
\maketitle
\vspace{0.5cm}
\tableofcontents
\clearpage

%*********************
\section{marked}
%---------------------
\begin{tcolorbox}[colback=white!10!white,colframe=lightgray!75!black,
  savelowerto=\jobname_ex.tex,breakable,enhanced,lines before break=40]

\justifying
Erläutern Sie ganz allgemein den Begriff Lokales Netz bzw. LAN. Welche grundlegende Funktionen bzw. Aufgabe erfüllt so ein LAN?

\tcblower

\justifying
LAN-Technologien realisieren den direkten physikalischen Datentransportzwischen einzelnen Netzinterfaces über ein sie verbindendes Medium.Ein LAN stellt die technische Infrastruktur zur Punkt-zu-Punkt Kommunikation über ein gemeinsames Medium direkt verbundene Systeme (Interfaces) dar.Die LAN-Technologien umfassen die OSI-Funktionsschichten -1 und -2.\begin{center}
\includegraphics[max width=.9\textwidth]{paste-14710262988801.jpg}
\end{center}
\begin{center}
\includegraphics[max width=.9\textwidth]{paste-14731737825281.jpg}
\end{center}
\begin{center}
\includegraphics[max width=.9\textwidth]{paste-58579058950145.png}
\end{center}
\begin{center}
\includegraphics[max width=.9\textwidth]{paste-58999965745153.png}
\end{center}

\end{tcolorbox}

\end{document}


