\documentclass{article}
\usepackage{tikz,lipsum,lmodern}
\usepackage[most]{tcolorbox}
\usepackage[paperheight=10.75in,paperwidth=7.25in,margin=1in,heightrounded]{geometry}
\usepackage{graphicx}
\usepackage{blindtext}
\usepackage{ragged2e}
\usepackage[space]{grffile}
\usepackage[utf8]{inputenc}

\graphicspath{{"/home/silasUser/.local/share/Anki2/User 1/collection.media/"}}

\title{todo extract deck name}

\begin{document}
%*********************
\section{Generelles}
%---------------------
\begin{tcolorbox}[colback=white!10!white,colframe=lightgray!75!black,
  savelowerto=\jobname_ex.tex]

\begin{center}
 Erläutern Sie die Motivation hinter der Datenbanktheorie. Gehen Sie insbesondere auf die 
Probleme
 ein, und beschreiben Sie was es heißt wenn ein Datum 
integer
ist. 

\end{center}

\tcblower

\justifying
\includegraphics[width=.9\textwidth]{paste-3440268804097.jpg}
\end{tcolorbox}
%---------------------
\begin{tcolorbox}[colback=white!10!white,colframe=lightgray!75!black,
  savelowerto=\jobname_ex.tex]

\begin{center}
 Erläutern Sie das 
ANSI-SPARC
Schema. Welches Ziel wird hierbei verfolgt. 

\end{center}

\tcblower

\justifying
\includegraphics[width=.9\textwidth]{paste-4144643440641.jpg}
\end{tcolorbox}
%---------------------
\begin{tcolorbox}[colback=white!10!white,colframe=lightgray!75!black,
  savelowerto=\jobname_ex.tex]

\begin{center}
 Wie werden Daten auf einer Magnetplatte 
adressiert?
Nennen Sie sowohl den Überbegriff als auch die einzelnen Komponenten aus denen sich dieser zusammensetzt. 

\end{center}

\tcblower

\justifying
\includegraphics[width=.9\textwidth]{paste-11072425689089.jpg}
\end{tcolorbox}
%---------------------
\begin{tcolorbox}[colback=white!10!white,colframe=lightgray!75!black,
  savelowerto=\jobname_ex.tex]

\begin{center}
 Wägen Sie ab warum es sinnvoll sein kann bei der Speicherung von Daten auf einer Magnetplatte auf größere Datenblöcke zu setzen. 

\end{center}

\tcblower

\justifying
Bei zu kleinen Datenblöcken müssen die zu speichernden Daten öfter aufgeteilt werden. Da diese im Zweifelsfall über die gesamte Platte verteilt sein können wird viel Zeit für diverse Schreib- und Lesevorgänge benötigt.

\end{tcolorbox}
%---------------------
\begin{tcolorbox}[colback=white!10!white,colframe=lightgray!75!black,
  savelowerto=\jobname_ex.tex]

\begin{center}
Bei zu großen Datenblöcken kann es vorkommen, dass eine kleines Datum (z.B. eine Zahl) nur einen geringen Teil der verfügbaren Blockgröße verwendet.

\end{center}

\tcblower

\justifying

\end{tcolorbox}
%---------------------
\begin{tcolorbox}[colback=white!10!white,colframe=lightgray!75!black,
  savelowerto=\jobname_ex.tex]

\begin{center}
 Erläutern Sie grob wie ein Raid-System funktioniert. Was versteht man unter dem sog. 
Striping
sowie dem 
Mirroring
? 

\end{center}

\tcblower

\justifying
\includegraphics[width=.9\textwidth]{paste-12695923326977.jpg}
\end{tcolorbox}
%---------------------
\begin{tcolorbox}[colback=white!10!white,colframe=lightgray!75!black,
  savelowerto=\jobname_ex.tex]

\begin{center}
 Erläutern Sie was man unter einem 
Raid-4 System
versteht. Wie wird hierbei die Parität ermittelt? 

\end{center}

\tcblower

\justifying
\includegraphics[width=.9\textwidth]{paste-13030930776065.jpg}\includegraphics[width=.9\textwidth]{paste-13597866459137.jpg}
\end{tcolorbox}

\end{document}


