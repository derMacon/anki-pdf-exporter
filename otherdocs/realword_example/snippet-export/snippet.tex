\documentclass{article}
\usepackage{tikz,lipsum,lmodern}
\usepackage[most]{tcolorbox}
\usepackage[paperheight=10.75in,paperwidth=7.25in,margin=1in,heightrounded]{geometry}
\usepackage{graphicx}
\usepackage{blindtext}
\usepackage{ragged2e}
\usepackage[space]{grffile}
\usepackage[utf8]{inputenc}

\graphicspath{{"/home/silasUser/.local/share/Anki2/User 1/collection.media/"}}

\title{todo extract deck name}

\begin{document}
%*********************
\section{Generelles}
%---------------------
\begin{tcolorbox}[colback=white!10!white,colframe=lightgray!75!black,
  savelowerto=\jobname_ex.tex]

\begin{center}
 Erläutern Sie die Motivation hinter der Datenbanktheorie. Gehen Sie insbesondere auf die 
Probleme
 ein, und beschreiben Sie was es heißt wenn ein Datum 
integer
ist. 

\end{center}

\tcblower

\justifying
\includegraphics[width=.9\textwidth]{paste-3440268804097.jpg}

\end{tcolorbox}
%---------------------
\begin{tcolorbox}[colback=white!10!white,colframe=lightgray!75!black,
  savelowerto=\jobname_ex.tex]

\begin{center}
 Erläutern Sie das 
ANSI-SPARC
Schema. Welches Ziel wird hierbei verfolgt. 

\end{center}

\tcblower

\justifying
\includegraphics[width=.9\textwidth]{paste-4144643440641.jpg}

\end{tcolorbox}
%---------------------
\begin{tcolorbox}[colback=white!10!white,colframe=lightgray!75!black,
  savelowerto=\jobname_ex.tex]

\begin{center}
 Wie werden Daten auf einer Magnetplatte 
adressiert?
Nennen Sie sowohl den Überbegriff als auch die einzelnen Komponenten aus denen sich dieser zusammensetzt. 

\end{center}

\tcblower

\justifying
\includegraphics[width=.9\textwidth]{paste-11072425689089.jpg}

\end{tcolorbox}
%---------------------
\begin{tcolorbox}[colback=white!10!white,colframe=lightgray!75!black,
  savelowerto=\jobname_ex.tex]

\begin{center}
 Wägen Sie ab warum es sinnvoll sein kann bei der Speicherung von Daten auf einer Magnetplatte auf größere Datenblöcke zu setzen. 

\end{center}

\tcblower

\justifying
Bei zu kleinen Datenblöcken müssen die zu speichernden Daten öfter aufgeteilt werden. Da diese im Zweifelsfall über die gesamte Platte verteilt sein können wird viel Zeit für diverse Schreib- und Lesevorgänge benötigt.

\end{tcolorbox}
%---------------------
\begin{tcolorbox}[colback=white!10!white,colframe=lightgray!75!black,
  savelowerto=\jobname_ex.tex]

\begin{center}
Bei zu großen Datenblöcken kann es vorkommen, dass eine kleines Datum (z.B. eine Zahl) nur einen geringen Teil der verfügbaren Blockgröße verwendet.

\end{center}

\tcblower

\justifying

\end{tcolorbox}
%---------------------
\begin{tcolorbox}[colback=white!10!white,colframe=lightgray!75!black,
  savelowerto=\jobname_ex.tex]

\begin{center}
 Erläutern Sie grob wie ein Raid-System funktioniert. Was versteht man unter dem sog. 
Striping
sowie dem 
Mirroring
? 

\end{center}

\tcblower

\justifying
\includegraphics[width=.9\textwidth]{paste-12695923326977.jpg}

\end{tcolorbox}
%---------------------
\begin{tcolorbox}[colback=white!10!white,colframe=lightgray!75!black,
  savelowerto=\jobname_ex.tex]

\begin{center}
 Erläutern Sie was man unter einem 
Raid-4 System
versteht. Wie wird hierbei die Parität ermittelt? 

\end{center}

\tcblower

\justifying
\includegraphics[width=.9\textwidth]{paste-13030930776065.jpg}
\includegraphics[width=.9\textwidth]{paste-13597866459137.jpg}

\end{tcolorbox}
%---------------------
\begin{tcolorbox}[colback=white!10!white,colframe=lightgray!75!black,
  savelowerto=\jobname_ex.tex]

\begin{center}
 
\includegraphics[width=.9\textwidth]{paste-18150531792897.jpg}
\includegraphics[width=.9\textwidth]{paste-18163416694785.jpg}
 

\end{center}

\tcblower

\justifying
\includegraphics[width=.9\textwidth]{paste-18176301596673.jpg}

\end{tcolorbox}
%---------------------
\begin{tcolorbox}[colback=white!10!white,colframe=lightgray!75!black,
  savelowerto=\jobname_ex.tex]

\begin{center}
 Warum erhöht ein Raid 4 die Ausfallsicherheit, ersetzt aber kein Backup? 

\end{center}

\tcblower

\justifying
Manuelles Löschen führt weiterhin zum Datenverlust.

\end{tcolorbox}
%---------------------
\begin{tcolorbox}[colback=white!10!white,colframe=lightgray!75!black,
  savelowerto=\jobname_ex.tex]

\begin{center}
 Weswegen wird in der Datenbanktheorie eine möglichst effiziente Pufferverwaltung angestrebt? Beschreiben Sie grob wie so eine aussehen könnte. 

\end{center}

\tcblower

\justifying
\includegraphics[width=.9\textwidth]{paste-7584912244737.jpg}

\end{tcolorbox}
%---------------------
\begin{tcolorbox}[colback=white!10!white,colframe=lightgray!75!black,
  savelowerto=\jobname_ex.tex]

\begin{center}
 Was versteht man unter dem 
Mapping
von Speicheradressen? 

\end{center}

\tcblower

\justifying
Zuordnung von Speicheradressen auf Scheiben, Spuren und Sektoren.

\end{tcolorbox}
%---------------------
\begin{tcolorbox}[colback=white!10!white,colframe=lightgray!75!black,
  savelowerto=\jobname_ex.tex]

\begin{center}
\includegraphics[width=.9\textwidth]{paste-7791070674945.jpg}

\end{center}

\tcblower

\justifying

\end{tcolorbox}
%---------------------
\begin{tcolorbox}[colback=white!10!white,colframe=lightgray!75!black,
  savelowerto=\jobname_ex.tex]

\begin{center}
 Beschreiben Sie generell den Begriff der Speicherverwaltung. 

\end{center}

\tcblower

\justifying
\includegraphics[width=.9\textwidth]{paste-7786775707649.jpg}

\end{tcolorbox}
%---------------------
\begin{tcolorbox}[colback=white!10!white,colframe=lightgray!75!black,
  savelowerto=\jobname_ex.tex]

\begin{center}
 Beschreiben Sie grob den Begriff der Speicherzuordnung. Mit welchem Hilfsmittel ist es hierbei möglich die Datenblöcke zu lokalisieren? Müssen die Datenblöcke stets die gleich Länge aufweisen? 

\end{center}

\tcblower

\justifying
\includegraphics[width=.9\textwidth]{paste-8504035246081.jpg}

\end{tcolorbox}
%---------------------
\begin{tcolorbox}[colback=white!10!white,colframe=lightgray!75!black,
  savelowerto=\jobname_ex.tex]

\begin{center}
 Was sind die Ziele der 
Pufferverwaltung
? 

\end{center}

\tcblower

\justifying
\includegraphics[width=.9\textwidth]{paste-8598524526593.jpg}

\end{tcolorbox}
%---------------------
\begin{tcolorbox}[colback=white!10!white,colframe=lightgray!75!black,
  savelowerto=\jobname_ex.tex]

\begin{center}
 Wo werden die Pufferdaten zwischengelagert? 

\end{center}

\tcblower

\justifying
In sog. 
Seiten
im Hauptspeicher (RAM).

\end{tcolorbox}
%---------------------
\begin{tcolorbox}[colback=white!10!white,colframe=lightgray!75!black,
  savelowerto=\jobname_ex.tex]

\begin{center}
\includegraphics[width=.9\textwidth]{paste-8594229559297.jpg}

\end{center}

\tcblower

\justifying

\end{tcolorbox}
%---------------------
\begin{tcolorbox}[colback=white!10!white,colframe=lightgray!75!black,
  savelowerto=\jobname_ex.tex]

\begin{center}
 Beschreiben Sie anhand der Datenbanken vom Hersteller 
Oracle
was man unter einem 
Segment 
sowie einem 
Extent 
versteht. Wie kann es sein, dass die Größe des 
Extents 
von den kleinsten logischen Einheiten einer Festplatte / SSD abweichen kann? 

\end{center}

\tcblower

\justifying
\includegraphics[width=.9\textwidth]{paste-9783935500289.jpg}
Da eine Datenbank i.d.R. ein eigenes Filesystem anlegt ist es komplett losgelöst von sämtlichen Restriktionen von Festspeicher / Betriebssystem.

\end{tcolorbox}
%---------------------
\begin{tcolorbox}[colback=white!10!white,colframe=lightgray!75!black,
  savelowerto=\jobname_ex.tex]

\begin{center}
 Was ermöglicht das Konzept eines 
Segments
in der Datenbanktheorie? 

\end{center}

\tcblower

\justifying
\includegraphics[width=.9\textwidth]{paste-10269266804737.jpg}

\end{tcolorbox}
%---------------------
\begin{tcolorbox}[colback=white!10!white,colframe=lightgray!75!black,
  savelowerto=\jobname_ex.tex]

\begin{center}
 Beschreiben Sie das Prinzip der 
direkten
Seitenadressierung an einem Bild. 

\end{center}

\tcblower

\justifying
\includegraphics[width=.9\textwidth]{paste-10690173599745.jpg}

\end{tcolorbox}
%---------------------
\begin{tcolorbox}[colback=white!10!white,colframe=lightgray!75!black,
  savelowerto=\jobname_ex.tex]

\begin{center}
 Beschreiben Sie das Prinzip der 
indirekten
Speicheradressierung. Gehen Sie insbesondere auf die beiden verwendeten Hilfsstrukturen ein. 

\end{center}

\tcblower

\justifying
\includegraphics[width=.9\textwidth]{paste-11016591114241.jpg}

\end{tcolorbox}
%---------------------
\begin{tcolorbox}[colback=white!10!white,colframe=lightgray!75!black,
  savelowerto=\jobname_ex.tex]

\begin{center}
 Beschreiben Sie das Prinzip der 
direkten
Einbringunsstrategie.
Welche Risiken können hierbei auftreten und wie kann man diesen entgegenwirken? 

\end{center}

\tcblower

\justifying
\includegraphics[width=.9\textwidth]{paste-11768210391041.jpg}
WAL-Prinzip: Man schreibt einen Log über die Dinge die als nächstes getan werden sollen (wie eine Todo-Liste). Falls hierbei ein Stromausfall dazwischen kommt weiß man genau welche Daten noch nicht kopiert wurden / noch gelöscht werden müssen um die Daten konsistent zu halten.

\end{tcolorbox}
%---------------------
\begin{tcolorbox}[colback=white!10!white,colframe=lightgray!75!black,
  savelowerto=\jobname_ex.tex]

\begin{center}
 Beschreiben Sie den Begriff der
verzögerten Einbringungsstrategie
am Beispiel des 
Schattenspeicherkonzepts.
 

\end{center}

\tcblower

\justifying
\includegraphics[width=.9\textwidth]{paste-12545599471617.jpg}
\includegraphics[width=.9\textwidth]{paste-12725988098049.jpg}
\includegraphics[width=.9\textwidth]{paste-12738872999937.jpg}

\end{tcolorbox}
%---------------------
\begin{tcolorbox}[colback=white!10!white,colframe=lightgray!75!black,
  savelowerto=\jobname_ex.tex]

\begin{center}
 Erläutern Sie das Konzept des 
Schattenspeicherkonzepts.
 

\end{center}

\tcblower

\justifying
\includegraphics[width=.9\textwidth]{paste-6648609374209.jpg}
\includegraphics[width=.9\textwidth]{paste-6661494276097.jpg}

\end{tcolorbox}
%---------------------
\begin{tcolorbox}[colback=white!10!white,colframe=lightgray!75!black,
  savelowerto=\jobname_ex.tex]

\begin{center}
 Erläutern Sie wie die Verwaltung des Puffers funktioniert? Gehen Sie insbesondere auf die Funktion des Pufferrahmens sowie des Pufferkontrollblocks ein. 

\end{center}

\tcblower

\justifying
\includegraphics[width=.9\textwidth]{paste-8693013807105.jpg}

\end{tcolorbox}
%---------------------
\begin{tcolorbox}[colback=white!10!white,colframe=lightgray!75!black,
  savelowerto=\jobname_ex.tex]

\begin{center}
 Welche Kontroll- und Zuteilungsaufgaben besitzt die Speicherverwaltung einer Datenbank? 

\end{center}

\tcblower

\justifying
\includegraphics[width=.9\textwidth]{paste-9745280794625.jpg}

\end{tcolorbox}
%---------------------
\begin{tcolorbox}[colback=white!10!white,colframe=lightgray!75!black,
  savelowerto=\jobname_ex.tex]

\begin{center}
 Gehen Sie auf die verschieden Szenarien ein welche beim Arbeiten mit einem Puffer auftreten können. 

\end{center}

\tcblower

\justifying
\includegraphics[width=.9\textwidth]{paste-10861972291585.jpg}

\end{tcolorbox}
%---------------------
\begin{tcolorbox}[colback=white!10!white,colframe=lightgray!75!black,
  savelowerto=\jobname_ex.tex]

\begin{center}
 Geben Sie eine grobe Übersicht über die verschiedenen Suchstrategien der Pufferverwaltung. 

\end{center}

\tcblower

\justifying
\includegraphics[width=.9\textwidth]{paste-11781095292929.jpg}

\end{tcolorbox}
%---------------------
\begin{tcolorbox}[colback=white!10!white,colframe=lightgray!75!black,
  savelowerto=\jobname_ex.tex]

\begin{center}
 Erläutern Sie nun noch kurz, wozu bei einem Router (oder einer Arbeitsstation) eine Wegwahl- bzw. Routingtabelle eigentlich dient. Was ist das bzw. was wird hier prinzipielle gespeichert? 

\end{center}

\tcblower

\justifying
In paketvermittelten Netzen werden Ende-zu-Ende-Verbindungen zwischen zwei Endgeräten dadurch hergestellt, dass Datenpakete von einer sendenden Station zu einer Empfangsstation übertragen werden. Die Stationen müssen über ihre Netzwerkadressen eindeutig identifizierbar sein. Zwischen den beiden Stationen liegen Router, die die Datenpakete gemäß ihrer Routingtabellen von der Sendestation über den ersten Router zum zweiten, zum dritten usw. weiterleiten, bis der letzte Router auf dem Weg durch das Datenpaketnetz die Datenpakete an die Empfängerstation ausliefert.

\end{tcolorbox}
%---------------------
\begin{tcolorbox}[colback=white!10!white,colframe=lightgray!75!black,
  savelowerto=\jobname_ex.tex]

\begin{center}
Routingtabellen werden von den Routing-Protokollen für die Pfadermittlung erstellt und können je nach Routing-Protokoll unterschiedlich sein. Beim statischen Routing ist der Routingpfad fest vorgegeben. Beim dynamischen Routing werden die Routingtabellen durch aktuelle Routinginformationen aktualisiert. Die Aktualisierung übernehmen Routing-Protokolle, die die Informationen eintragen. Das können Information über die Verknüpfungen der Netzwerke, bzw. der Endgeräte und der Hops mit den dazu gehörenden IP-Adressen sein, oder die Routing-Metriken mit der die Route festgelegt wurde.

\end{center}

\tcblower

\justifying

\end{tcolorbox}
%---------------------
\begin{tcolorbox}[colback=white!10!white,colframe=lightgray!75!black,
  savelowerto=\jobname_ex.tex]

\begin{center}
\includegraphics[width=.9\textwidth]{paste-3264175144961.png}

\end{center}

\tcblower

\justifying

\end{tcolorbox}
%---------------------
\begin{tcolorbox}[colback=white!10!white,colframe=lightgray!75!black,
  savelowerto=\jobname_ex.tex]

\begin{center}
 Was versteht man unter einem 
logischen
Seitenreferenzstring? 

\end{center}

\tcblower

\justifying
\includegraphics[width=.9\textwidth]{paste-4050154160129.jpg}
\includegraphics[width=.9\textwidth]{paste-4080218931201.jpg}

\end{tcolorbox}
%---------------------
\begin{tcolorbox}[colback=white!10!white,colframe=lightgray!75!black,
  savelowerto=\jobname_ex.tex]

\begin{center}
 Geben Sie eine kurze Aufschlüsselung über die möglichen Suchstrategien bei der 
Pufferverwaltung.
 

\end{center}

\tcblower

\justifying

\end{tcolorbox}
%---------------------
\begin{tcolorbox}[colback=white!10!white,colframe=lightgray!75!black,
  savelowerto=\jobname_ex.tex]

\begin{center}
 Erläutern Sie die 
Vorgehensweise 
bei der direkten Suche im Pufferrahmen, wie viele Rahmen werden durchschnittlich durchsucht und was ist der wesentliche Nachteil dieser Technik? 

\end{center}

\tcblower

\justifying
\includegraphics[width=.9\textwidth]{paste-4780298600449.jpg}

\end{tcolorbox}
%---------------------
\begin{tcolorbox}[colback=white!10!white,colframe=lightgray!75!black,
  savelowerto=\jobname_ex.tex]

\begin{center}
 Erläutern Sie wie eine indirekten Suche mittels einer Hashfunktion funktioniert. Gehen Sie hierbei insbesondere auf eine sog. Überlaufkette ein. 

\end{center}

\tcblower

\justifying
\includegraphics[width=.9\textwidth]{paste-3809635991553.png}
\includegraphics[width=.9\textwidth]{paste-3831110828033.png}

\end{tcolorbox}
%---------------------
\begin{tcolorbox}[colback=white!10!white,colframe=lightgray!75!black,
  savelowerto=\jobname_ex.tex]

\begin{center}
 Wie funktioniert die Suche nach den im Puffer über eine 
direkt adressierte 
Umsetzungstabelle
? Welchen wesentlichen Nachteil bringt dieses Verfahren mit sich? 

\end{center}

\tcblower

\justifying
\includegraphics[width=.9\textwidth]{paste-4913442586625 (1).png}

\end{tcolorbox}
%---------------------
\begin{tcolorbox}[colback=white!10!white,colframe=lightgray!75!black,
  savelowerto=\jobname_ex.tex]

\begin{center}
 Wie funktioniert die indirekte Suche nach einer Seite im Pufferrahmen mittels einer 
unsortierten
 Liste?Gehen Sie insbesondere auf die Vor-/Nachteile sowie die Laufzeit des Verfahrens ein. 

\end{center}

\tcblower

\justifying
\includegraphics[width=.9\textwidth]{paste-5862630359041.png}

\end{tcolorbox}
%---------------------
\begin{tcolorbox}[colback=white!10!white,colframe=lightgray!75!black,
  savelowerto=\jobname_ex.tex]

\begin{center}
 Wie funktioniert die indirekte Suche nach einer Seite im Pufferrahmen mittels einer
sortierten
 Liste? Gehen Sie insbesondere auf die Vor-/Nachteile sowie die Laufzeit des Verfahrens ein. 

\end{center}

\tcblower

\justifying
\includegraphics[width=.9\textwidth]{paste-5935644803073.png}

\end{tcolorbox}
%---------------------
\begin{tcolorbox}[colback=white!10!white,colframe=lightgray!75!black,
  savelowerto=\jobname_ex.tex]

\begin{center}
 Wie funktioniert die indirekte Suche nach einer Seite im Pufferrahmen mittels einer
sortierten
mit verketteten Einträgen
Liste? Gehen Sie insbesondere auf die Vorteile des Verfahrens ein. 

\end{center}

\tcblower

\justifying
\includegraphics[width=.9\textwidth]{paste-7962869366785 (1).png}

\end{tcolorbox}

\end{document}


