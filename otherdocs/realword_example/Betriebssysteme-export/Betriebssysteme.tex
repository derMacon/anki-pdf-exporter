\documentclass{article}
\usepackage{tikz,lipsum,lmodern}
\usepackage[ngerman]{babel}
\usepackage[most]{tcolorbox}
\usepackage[paperheight=10.75in,paperwidth=7.25in,margin=1in,heightrounded]{geometry}
\usepackage{graphicx}
\usepackage{blindtext}
\usepackage{ragged2e}
\usepackage{needspace}
\usepackage[space]{grffile}
\usepackage[utf8]{inputenc}
\usepackage[export]{adjustbox}
\usepackage{hyperref}
\hypersetup{
    colorlinks,
    citecolor=black,
    filecolor=black,
    linkcolor=black,
    urlcolor=black
}
\usepackage{fancyhdr}
\pagestyle{fancy}
\fancyhf{}
\rhead{\rightmark}
\chead{\thepart}
\lhead{\nouppercase{\leftmark}}
\cfoot{\thepage}
\graphicspath{{"/home/silasUser/.local/share/Anki2/User 1/collection.media/"}}

\title{Betriebssysteme}
\author{Silas Hoffmann, inf103088}
\date{\today}



\begin{document}
\maketitle
\vspace{0.5cm}
\tableofcontents
\clearpage

%*********************
\Needspace{20\baselineskip}
\section{Betriebssysteme}
\Needspace{20\baselineskip}
\subsection{Einführung}
%---------------------
\begin{tcolorbox}[colback=white!10!white,colframe=lightgray!75!black,
  savelowerto=\jobname_ex.tex,breakable,enhanced,lines before break=40]

\justifying
Charakterisieren Sie das BIOS und nennen Sie die wichtigsten Aufgaben. Wofuer steht die Abkuerzung ueberhaupt?

\tcblower

\justifying
\begin{center}
\includegraphics[max width=.9\textwidth]{betriebssysteme_skript_12.png}
\end{center}

\end{tcolorbox}
%---------------------
\begin{tcolorbox}[colback=white!10!white,colframe=lightgray!75!black,
  savelowerto=\jobname_ex.tex,breakable,enhanced,lines before break=40]

\justifying
Richtig oder falsch: Ältere Betriebssysteme wie MS-DOS benutzten Basisfunktionen des BIOS auch nach dem Laden des eigentlichen BS.

\tcblower

\justifying
Richtig. Neuere Varianten tun dies nicht mehr, sodass man gar den Chip des BIOS während des laufenden Betriebs entfernen kann.\begin{center}
\includegraphics[max width=.9\textwidth]{paste-17304423235585.png}
\end{center}

\end{tcolorbox}
%---------------------
\begin{tcolorbox}[colback=white!10!white,colframe=lightgray!75!black,
  savelowerto=\jobname_ex.tex,breakable,enhanced,lines before break=40]

\justifying
Erläutern Sie den Begriff des \textbf{UEFI}wofür steht diese Abkürzung, woher kommt es und welche Vorteile bietet es im Gegensatz zum BIOS?Erläutern Sie ebenfalls kurz was man unter dem Begriff \textit{Firmware }versteht.Welche Nachteile bietet das Booten mittels UEFI?

\tcblower

\justifying
\begin{center}
\includegraphics[max width=.9\textwidth]{paste-17836999180289 (1).png}
\end{center}

\end{tcolorbox}
%---------------------
\begin{tcolorbox}[colback=white!10!white,colframe=lightgray!75!black,
  savelowerto=\jobname_ex.tex,breakable,enhanced,lines before break=40]

\justifying
Skizzieren Sie den \textbf{Bootvorgang} mittels UEFI und grenzen Sie diesen von dem mittels des klassischen BIOS ab.

\tcblower

\justifying
\begin{center}
\includegraphics[max width=.9\textwidth]{paste-21775484190721.png}
\end{center}

\end{tcolorbox}
%---------------------
\begin{tcolorbox}[colback=white!10!white,colframe=lightgray!75!black,
  savelowerto=\jobname_ex.tex,breakable,enhanced,lines before break=40]

\justifying
Erläutern Sie die BS\textit{Dienstleistung}der \textbf{Abstraktion}von konkreter Hardware. Wieso ist so etwas sinnvoll und gibt es hierbei Ausnahmen?

\tcblower

\justifying
\begin{center}
\includegraphics[max width=.9\textwidth]{paste-22213570854913.png}
\end{center}

\end{tcolorbox}
%---------------------
\begin{tcolorbox}[colback=white!10!white,colframe=lightgray!75!black,
  savelowerto=\jobname_ex.tex,breakable,enhanced,lines before break=40]

\justifying
Welche Dienstleistungen stellt ein Betriebssystem im Bereich der \textbf{Speichermedien} bereit?

\tcblower

\justifying
\begin{center}
\includegraphics[max width=.9\textwidth]{paste-22617297780737.png}
\end{center}

\end{tcolorbox}
%---------------------
\begin{tcolorbox}[colback=white!10!white,colframe=lightgray!75!black,
  savelowerto=\jobname_ex.tex,breakable,enhanced,lines before break=40]

\justifying
Welche Dienstleistungen stellt ein BS im Bereich der \textbf{Prozessverwaltung}bereit? Erläutern Sie ebenfalls diesen Begriff.

\tcblower

\justifying
\begin{center}
\includegraphics[max width=.9\textwidth]{paste-22965190131713.png}
\end{center}

\end{tcolorbox}
%---------------------
\begin{tcolorbox}[colback=white!10!white,colframe=lightgray!75!black,
  savelowerto=\jobname_ex.tex,breakable,enhanced,lines before break=40]

\justifying
Nennen und beschreiben Sie alle wesentlichen Dienstleistungen die ein Betriebssystem zur Verfügung stellt.

\tcblower

\justifying
\begin{center}
\includegraphics[max width=.9\textwidth]{paste-23287312678913.png}
\end{center}
\begin{center}
\includegraphics[max width=.9\textwidth]{paste-23300197580801.png}
\end{center}
\begin{center}
\includegraphics[max width=.9\textwidth]{paste-23313082482689.png}
\end{center}
\begin{center}
\includegraphics[max width=.9\textwidth]{paste-22960895164417.png}
\end{center}

\end{tcolorbox}
%---------------------
\begin{tcolorbox}[colback=white!10!white,colframe=lightgray!75!black,
  savelowerto=\jobname_ex.tex,breakable,enhanced,lines before break=40]

\justifying
Beschreiben Sie grob die Marktanteile der PC-Betriebssysteme.

\tcblower

\justifying
\begin{center}
\includegraphics[max width=.9\textwidth]{paste-23476291239937.png}
\end{center}
\begin{center}
\includegraphics[max width=.9\textwidth]{paste-23489176141825 (1).png}
\end{center}
\begin{center}
\includegraphics[max width=.9\textwidth]{paste-23510650978305 (1).png}
\end{center}

\end{tcolorbox}
%---------------------
\begin{tcolorbox}[colback=white!10!white,colframe=lightgray!75!black,
  savelowerto=\jobname_ex.tex,breakable,enhanced,lines before break=40]

\justifying
Beschreiben Sie grob die Marktanteile der \textbf{mobilen-Betriebssysteme}.<missing IDENTIFIER>

\tcblower

\justifying
\begin{center}
\includegraphics[max width=.9\textwidth]{paste-23708219473921.png}
\end{center}
\begin{center}
\includegraphics[max width=.9\textwidth]{paste-23721104375809.png}
\end{center}

\end{tcolorbox}
%---------------------
\begin{tcolorbox}[colback=white!10!white,colframe=lightgray!75!black,
  savelowerto=\jobname_ex.tex,breakable,enhanced,lines before break=40]

\justifying
Nennen Sie die Besonderheiten von \textbf{Server-Betriebssystemen}.<missing IDENTIFIER>

\tcblower

\justifying
\begin{center}
\includegraphics[max width=.9\textwidth]{paste-23794118819841.png}
\end{center}

\end{tcolorbox}
%---------------------
\begin{tcolorbox}[colback=white!10!white,colframe=lightgray!75!black,
  savelowerto=\jobname_ex.tex,breakable,enhanced,lines before break=40]

\justifying
Nennen Sie die Besonderheiten von \textbf{Echtzeitbetriebssystemen}. Und nennen Sie einige Beispiele.

\tcblower

\justifying
\begin{center}
\includegraphics[max width=.9\textwidth]{paste-23935852740609.png}
\end{center}
\begin{center}
\includegraphics[max width=.9\textwidth]{paste-23948737642497.png}
\end{center}

\end{tcolorbox}
%---------------------
\begin{tcolorbox}[colback=white!10!white,colframe=lightgray!75!black,
  savelowerto=\jobname_ex.tex,breakable,enhanced,lines before break=40]

\justifying
Fassen Sie die verschiedenen Arten von Betriebssystemen zusammen.

\tcblower

\justifying
\begin{center}
\includegraphics[max width=.9\textwidth]{paste-24163486007297.png}
\end{center}
\begin{center}
\includegraphics[max width=.9\textwidth]{paste-24146306138113.png}
\end{center}

\end{tcolorbox}
%---------------------
\begin{tcolorbox}[colback=white!10!white,colframe=lightgray!75!black,
  savelowerto=\jobname_ex.tex,breakable,enhanced,lines before break=40]

\justifying
Was ist ein \textbf{Kernel}und woraus bestehen seine wesentlichen Aufgaben?

\tcblower

\justifying
\textbf{Zusammenfassung - Kernel:}Software Kern des Betriebssystems, ist selbst ein Programm welches Dienste bereitstellt.\begin{center}
\includegraphics[max width=.9\textwidth]{paste-24313809862657.png}
\end{center}

\end{tcolorbox}
%---------------------
\begin{tcolorbox}[colback=white!10!white,colframe=lightgray!75!black,
  savelowerto=\jobname_ex.tex,breakable,enhanced,lines before break=40]

\justifying
Nennen Sie die Hauptarten von BS-Kernel.

\tcblower

\justifying
\begin{center}
\includegraphics[max width=.9\textwidth]{paste-24902220382209 (1).png}
\end{center}

\end{tcolorbox}
%---------------------
\begin{tcolorbox}[colback=white!10!white,colframe=lightgray!75!black,
  savelowerto=\jobname_ex.tex,breakable,enhanced,lines before break=40]

\justifying
Beschreiben Sie die \textit{Philosophie, Nachteile}und die \textit{Vorteile}eines \textbf{monolithischen Kernel BS}.<missing IDENTIFIER>

\tcblower

\justifying
\begin{center}
\includegraphics[max width=.9\textwidth]{paste-25082609008641 (1).png}
\end{center}

\end{tcolorbox}
%---------------------
\begin{tcolorbox}[colback=white!10!white,colframe=lightgray!75!black,
  savelowerto=\jobname_ex.tex,breakable,enhanced,lines before break=40]

\justifying
Beschreiben Sie die\textit{Philosophie, Nachteile}und die\textit{Vorteile}eines\textbf{Mikrokernel BS}.<missing IDENTIFIER>

\tcblower

\justifying
\begin{center}
\includegraphics[max width=.9\textwidth]{paste-25232932864001.png}
\end{center}

\end{tcolorbox}
%---------------------
\begin{tcolorbox}[colback=white!10!white,colframe=lightgray!75!black,
  savelowerto=\jobname_ex.tex,breakable,enhanced,lines before break=40]

\justifying
Beschreiben Sie die\textit{Philosophie, Nachteile}und die\textit{Vorteile}eines \textbf{hybridenBS Kernels}.<missing IDENTIFIER>

\tcblower

\justifying
\begin{center}
\includegraphics[max width=.9\textwidth]{paste-25701084299265 (1).png}
\end{center}

\end{tcolorbox}
%---------------------
\begin{tcolorbox}[colback=white!10!white,colframe=lightgray!75!black,
  savelowerto=\jobname_ex.tex,breakable,enhanced,lines before break=40]

\justifying
Fassen Sie noch einmal die verschiedenen Klassen von Kernels zusammen und grenzen Sie diese voneinander ab.

\tcblower

\justifying
\begin{center}
\includegraphics[max width=.9\textwidth]{paste-25756918874113.png}
\end{center}
\begin{center}
\includegraphics[max width=.9\textwidth]{paste-25769803776001.png}
\end{center}
\begin{center}
\includegraphics[max width=.9\textwidth]{paste-25786983645185.png}
\end{center}
\begin{center}
\includegraphics[max width=.9\textwidth]{paste-25799868547073.png}
\end{center}

\end{tcolorbox}
%---------------------
\begin{tcolorbox}[colback=white!10!white,colframe=lightgray!75!black,
  savelowerto=\jobname_ex.tex,breakable,enhanced,lines before break=40]

\justifying
Nennen Sie die wesentlichen Vorteile von \textbf{UEFI} gegenüber \textbf{BIOS}!<missing IDENTIFIER>

\tcblower

\justifying
\begin{center}
\includegraphics[max width=.9\textwidth]{betriebssysteme_skript_49.png}
\end{center}

\end{tcolorbox}
%---------------------
\begin{tcolorbox}[colback=white!10!white,colframe=lightgray!75!black,
  savelowerto=\jobname_ex.tex,breakable,enhanced,lines before break=40]

\justifying
Nennen Sie die wesentlichen Dienstleistungen eines Betriebssystems

\tcblower

\justifying
\begin{center}
\includegraphics[max width=.9\textwidth]{betriebssysteme_skript_49.png}
\end{center}

\end{tcolorbox}
%---------------------
\begin{tcolorbox}[colback=white!10!white,colframe=lightgray!75!black,
  savelowerto=\jobname_ex.tex,breakable,enhanced,lines before break=40]

\justifying
Welche der Kernel Aufgaben gehört \textbf{NICHT} zu den Aufgaben eines BS-Kerns?

\tcblower

\justifying
\begin{center}
\includegraphics[max width=.9\textwidth]{betriebssysteme_skript_50.png}
\end{center}

\end{tcolorbox}
%---------------------
\begin{tcolorbox}[colback=white!10!white,colframe=lightgray!75!black,
  savelowerto=\jobname_ex.tex,breakable,enhanced,lines before break=40]

\justifying
Beschreiben Sie die drei Hauptarten von BS-Kernel und gehen Sie dabei auf die Vor- und Nachteile ein!

\tcblower

\justifying
\begin{center}
\includegraphics[max width=.9\textwidth]{betriebssysteme_skript_50.png}
\end{center}

\end{tcolorbox}
\Needspace{20\baselineskip}
\subsection{ProzesseUndThreads}
%---------------------
\begin{tcolorbox}[colback=white!10!white,colframe=lightgray!75!black,
  savelowerto=\jobname_ex.tex,breakable,enhanced,lines before break=40]

\justifying
Definieren Sie den Begriff des \textbf{Prozesses}. Was versteht man in diesem Zusammenhang unter der der \textit{Quasiparallelität}?<missing IDENTIFIER>

\tcblower

\justifying
\begin{center}
\includegraphics[max width=.9\textwidth]{betriebssysteme_skript_55.png}
\end{center}
\begin{center}
\includegraphics[max width=.9\textwidth]{betriebssysteme_skript_54.png}
\end{center}

\end{tcolorbox}
%---------------------
\begin{tcolorbox}[colback=white!10!white,colframe=lightgray!75!black,
  savelowerto=\jobname_ex.tex,breakable,enhanced,lines before break=40]

\justifying
Was versteht man unter dem sog. \textbf{Prozessmodell}?<missing IDENTIFIER>Erläutern Sie in diesem Zusammenhang grob, was man unter einer \textit{Schedulingstrategie} versteht.

\tcblower

\justifying
\begin{center}
\includegraphics[max width=.9\textwidth]{betriebssysteme_skript_55.png}
\end{center}
\begin{center}
\includegraphics[max width=.9\textwidth]{betriebssysteme_skript_54.png}
\end{center}

\end{tcolorbox}
%---------------------
\begin{tcolorbox}[colback=white!10!white,colframe=lightgray!75!black,
  savelowerto=\jobname_ex.tex,breakable,enhanced,lines before break=40]

\justifying
Beschreiben Sie den Begriff der \textbf{Prozesstabelle}.<missing IDENTIFIER>Welche Daten werden hierbei festgehalten?

\tcblower

\justifying
\begin{center}
\includegraphics[max width=.9\textwidth]{betriebssysteme_skript_55.png}
\end{center}

\end{tcolorbox}
%---------------------
\begin{tcolorbox}[colback=white!10!white,colframe=lightgray!75!black,
  savelowerto=\jobname_ex.tex,breakable,enhanced,lines before break=40]

\justifying
Was versteht man unter einem \textit{Prozesskontrollblock}?<missing IDENTIFIER>

\tcblower

\justifying
\begin{center}
\includegraphics[max width=.9\textwidth]{betriebssysteme_skript_55.png}
\end{center}
\begin{center}
\includegraphics[max width=.9\textwidth]{betriebssysteme_skript_56.png}
\end{center}

\end{tcolorbox}
%---------------------
\begin{tcolorbox}[colback=white!10!white,colframe=lightgray!75!black,
  savelowerto=\jobname_ex.tex,breakable,enhanced,lines before break=40]

\justifying
Erläutern Sie den Begriff der \textbf{Prozesserzeugung}. Wann ist so etwas sinnvoll, welche Möglichkeiten der Erzeugung gibt es, welche Befehle werden verwendet?

\tcblower

\justifying
\begin{center}
\includegraphics[max width=.9\textwidth]{betriebssysteme_skript_57.png}
\end{center}

\end{tcolorbox}
%---------------------
\begin{tcolorbox}[colback=white!10!white,colframe=lightgray!75!black,
  savelowerto=\jobname_ex.tex,breakable,enhanced,lines before break=40]

\justifying
Wie kann ein Prozess beendet werden?Welche unterschiedlichen Arten gibt es, erläutern Sie dies am Beispiel vom BS Linux.

\tcblower

\justifying
\begin{center}
\includegraphics[max width=.9\textwidth]{betriebssysteme_skript_58.png}
\end{center}

\end{tcolorbox}
%---------------------
\begin{tcolorbox}[colback=white!10!white,colframe=lightgray!75!black,
  savelowerto=\jobname_ex.tex,breakable,enhanced,lines before break=40]

\justifying
Erläutern Sie die unterschiedlichen Prozesshierarchien von Windows und Unix-Maschinen.Wie können solche Prozesse erzeugt werden und wie sieht deren Beziehung untereinander aus.

\tcblower

\justifying
\begin{center}
\includegraphics[max width=.9\textwidth]{betriebssysteme_skript_59.png}
\end{center}

\end{tcolorbox}
%---------------------
\begin{tcolorbox}[colback=white!10!white,colframe=lightgray!75!black,
  savelowerto=\jobname_ex.tex,breakable,enhanced,lines before break=40]

\justifying
Erläutern Sie das grundsätzliche Zustandsmodell der möglichen Prozesszustände anhand einer Skizze.

\tcblower

\justifying
\begin{center}
\includegraphics[max width=.9\textwidth]{betriebssysteme_skript_60.png}
\end{center}

\end{tcolorbox}
%---------------------
\begin{tcolorbox}[colback=white!10!white,colframe=lightgray!75!black,
  savelowerto=\jobname_ex.tex,breakable,enhanced,lines before break=40]

\justifying
Erläutern Sie anhand eines einfachen Beispiels die Sinnhaftigkeit von Threads.

\tcblower

\justifying
\begin{center}
\includegraphics[max width=.9\textwidth]{betriebssysteme_skript_63.png}
\end{center}

\end{tcolorbox}
%---------------------
\begin{tcolorbox}[colback=white!10!white,colframe=lightgray!75!black,
  savelowerto=\jobname_ex.tex,breakable,enhanced,lines before break=40]

\justifying
Worin genau liegt der Unterschied zwischen \textit{Prozessen} und \textit{Threads}?<missing IDENTIFIER>

\tcblower

\justifying
\begin{center}
\includegraphics[max width=.9\textwidth]{betriebssysteme_skript_64.png}
\end{center}
\begin{center}
\includegraphics[max width=.9\textwidth]{betriebssysteme_skript_65.png}
\end{center}

\end{tcolorbox}
%---------------------
\begin{tcolorbox}[colback=white!10!white,colframe=lightgray!75!black,
  savelowerto=\jobname_ex.tex,breakable,enhanced,lines before break=40]

\justifying
Nennen Sie die beiden Möglichkeiten der Thread-erzeugung unter Linux. Worin liegt der Unterschied zur Erzeugung von Prozessen?

\tcblower

\justifying
\begin{center}
\includegraphics[max width=.9\textwidth]{betriebssysteme_skript_66.png}
\end{center}
\begin{center}
\includegraphics[max width=.9\textwidth]{betriebssysteme_skript_67.png}
\end{center}
\begin{center}
\includegraphics[max width=.9\textwidth]{betriebssysteme_skript_68.png}
\end{center}
\begin{center}
\includegraphics[max width=.9\textwidth]{betriebssysteme_skript_69.png}
\end{center}
\begin{center}
\includegraphics[max width=.9\textwidth]{betriebssysteme_skript_70.png}
\end{center}

\end{tcolorbox}
%---------------------
\begin{tcolorbox}[colback=white!10!white,colframe=lightgray!75!black,
  savelowerto=\jobname_ex.tex,breakable,enhanced,lines before break=40]

\justifying
Was versteht man unter einem sog. \textbf{Scheduler}?<missing IDENTIFIER>

\tcblower

\justifying
\begin{center}
\includegraphics[max width=.9\textwidth]{betriebssysteme_skript_72.png}
\end{center}
Das \textit{Kleingedruckte}: Permanent am Arbeiten, muss selber bei der CPU rechnend aktiv werden um anderen Prozessen ihre Zuteilung bereitzustellen.
\end{tcolorbox}
%---------------------
\begin{tcolorbox}[colback=white!10!white,colframe=lightgray!75!black,
  savelowerto=\jobname_ex.tex,breakable,enhanced,lines before break=40]

\justifying
Auf welchen Systemen ist ein \textbf{Scheduler} von besonderer Bedeutung? Gehen Sie hierbei insbesondere auf die Gründe ein.

\tcblower

\justifying
\begin{center}
\includegraphics[max width=.9\textwidth]{betriebssysteme_skript_73.png}
\end{center}

\end{tcolorbox}
%---------------------
\begin{tcolorbox}[colback=white!10!white,colframe=lightgray!75!black,
  savelowerto=\jobname_ex.tex,breakable,enhanced,lines before break=40]

\justifying
Erläutern Sie den Begriff des \textbf{Prozessverhaltens} im Zusammenhang mit dem \textit{Scheduler}.<missing IDENTIFIER>

\tcblower

\justifying
\begin{center}
\includegraphics[max width=.9\textwidth]{betriebssysteme_skript_74.png}
\end{center}

\end{tcolorbox}
%---------------------
\begin{tcolorbox}[colback=white!10!white,colframe=lightgray!75!black,
  savelowerto=\jobname_ex.tex,breakable,enhanced,lines before break=40]

\justifying
Welche Möglichkeiten gibt es \textit{wann} das Scheduling beginnen sollte?

\tcblower

\justifying
\begin{center}
\includegraphics[max width=.9\textwidth]{betriebssysteme_skript_74.png}
\end{center}
\begin{center}
\includegraphics[max width=.9\textwidth]{betriebssysteme_skript_75.png}
\end{center}

\end{tcolorbox}
%---------------------
\begin{tcolorbox}[colback=white!10!white,colframe=lightgray!75!black,
  savelowerto=\jobname_ex.tex,breakable,enhanced,lines before break=40]

\justifying
Grenzen Sie die \textit{unterbrechbaren} von den \textit{nicht-unterbrechbaren} Strategien ab.

\tcblower

\justifying
\begin{center}
\includegraphics[max width=.9\textwidth]{betriebssysteme_skript_75.png}
\end{center}

\end{tcolorbox}
%---------------------
\begin{tcolorbox}[colback=white!10!white,colframe=lightgray!75!black,
  savelowerto=\jobname_ex.tex,breakable,enhanced,lines before break=40]

\justifying
Nennen Sie die beiden Möglichkeiten der Thread-Erzeugung unter Linux. Worin liegt der Unterschied zur Erzeugung von Prozessen?

\tcblower

\justifying
\begin{center}
\includegraphics[max width=.9\textwidth]{betriebssysteme_skript_66.png}
\end{center}
\begin{center}
\includegraphics[max width=.9\textwidth]{betriebssysteme_skript_67.png}
\end{center}
\begin{center}
\includegraphics[max width=.9\textwidth]{betriebssysteme_skript_68.png}
\end{center}
\begin{center}
\includegraphics[max width=.9\textwidth]{betriebssysteme_skript_69.png}
\end{center}
\begin{center}
\includegraphics[max width=.9\textwidth]{betriebssysteme_skript_70.png}
\end{center}

\end{tcolorbox}
%---------------------
\begin{tcolorbox}[colback=white!10!white,colframe=lightgray!75!black,
  savelowerto=\jobname_ex.tex,breakable,enhanced,lines before break=40]

\justifying
Erläutern Sie die möglichen Kriterien für ein gutes \textbf{präemtives} Scheduling. Grenzen Sie dabei noch einmal die beiden wesentlichen \textit{Scheduling-Strategien} voneinander ab.

\tcblower

\justifying
\begin{center}
\includegraphics[max width=.9\textwidth]{betriebssysteme_skript_77.png}
\end{center}
\begin{center}
\includegraphics[max width=.9\textwidth]{betriebssysteme_skript_78.png}
\end{center}
\begin{center}
\includegraphics[max width=.9\textwidth]{betriebssysteme_skript_79.png}
\end{center}

\end{tcolorbox}
%---------------------
\begin{tcolorbox}[colback=white!10!white,colframe=lightgray!75!black,
  savelowerto=\jobname_ex.tex,breakable,enhanced,lines before break=40]

\justifying
Welche der Kriterien für ein gutes \textbf{präemtives} Scheduling schliessen sich gegenseitig aus? Nennen Sie zwei Beispiele.

\tcblower

\justifying
\begin{center}
\includegraphics[max width=.9\textwidth]{betriebssysteme_skript_79.png}
\end{center}

\end{tcolorbox}
%---------------------
\begin{tcolorbox}[colback=white!10!white,colframe=lightgray!75!black,
  savelowerto=\jobname_ex.tex,breakable,enhanced,lines before break=40]

\justifying
Erläutern Sie die Grundidee sowie die Vor- / Nachteile der \textbf{FCFS} Scheduling-Strategie. In welche Kategorie lässt sich diese einordnen? Wofür steht die \textit{Abkürzung}?<missing IDENTIFIER>Nennen Sie eine Situation aus dem realen Leben die mit dieser Strategie gut verglichen werden kann.

\tcblower

\justifying
\begin{center}
\includegraphics[max width=.9\textwidth]{betriebssysteme_skript_80.png}
\end{center}

\end{tcolorbox}
%---------------------
\begin{tcolorbox}[colback=white!10!white,colframe=lightgray!75!black,
  savelowerto=\jobname_ex.tex,breakable,enhanced,lines before break=40]

\justifying
Erläutern Sie die Grundidee sowie die Vor- / Nachteile der \textbf{SJF} Scheduling-Strategie. In welche Kategorie lässt sich diese einordnen? Wofür steht die \textit{Abkürzung}?<missing IDENTIFIER>Nennen Sie eine Situation aus dem realen Leben die mit dieser Strategie gut verglichen werden kann.

\tcblower

\justifying
\begin{center}
\includegraphics[max width=.9\textwidth]{betriebssysteme_skript_81.png}
\end{center}

\end{tcolorbox}
%---------------------
\begin{tcolorbox}[colback=white!10!white,colframe=lightgray!75!black,
  savelowerto=\jobname_ex.tex,breakable,enhanced,lines before break=40]

\justifying
Erläutern Sie die Grundidee sowie die Vor- / Nachteile des \textbf{Zeitschreibeverfahrens}. In welche Kategorie lässt sich diese Strategie einordnen?Nennen Sie eine Situation aus dem realen Leben die mit dieser Strategie gut verglichen werden kann.Unter welchem \textit{Namen} ist diese Strategie allgemeinhin noch bekannt?Nennen Sie ein Beispiel einer \textbf{Variation} dieses Verfahrens.

\tcblower

\justifying
\begin{center}
\includegraphics[max width=.9\textwidth]{betriebssysteme_skript_82.png}
\end{center}

\end{tcolorbox}
%---------------------
\begin{tcolorbox}[colback=white!10!white,colframe=lightgray!75!black,
  savelowerto=\jobname_ex.tex,breakable,enhanced,lines before break=40]

\justifying
Was versteht man im Zusammenhang des \textbf{Zeitschreibeverfahrens} unter einem \textit{Quantum}?<missing IDENTIFIER>

\tcblower

\justifying
\begin{center}
\includegraphics[max width=.9\textwidth]{betriebssysteme_skript_82.png}
\end{center}
\begin{center}
\includegraphics[max width=.9\textwidth]{betriebssysteme_skript_83.png}
\end{center}
\begin{center}
\includegraphics[max width=.9\textwidth]{betriebssysteme_skript_84.png}
\end{center}

\end{tcolorbox}
%---------------------
\begin{tcolorbox}[colback=white!10!white,colframe=lightgray!75!black,
  savelowerto=\jobname_ex.tex,breakable,enhanced,lines before break=40]

\justifying
Erläutern Sie den Begriff sowie die \textbf{Grundidee} des \textbf{MQS}, wofür steht diese \textit{Abkürzung}?<missing IDENTIFIER>

\tcblower

\justifying
\begin{center}
\includegraphics[max width=.9\textwidth]{betriebssysteme_skript_86.png}
\end{center}

\end{tcolorbox}
%---------------------
\begin{tcolorbox}[colback=white!10!white,colframe=lightgray!75!black,
  savelowerto=\jobname_ex.tex,breakable,enhanced,lines before break=40]

\justifying
Welche Rolle spielt der Begriff \textbf{Feedback} in der \textbf{Multilevel \textit{Feedback} Queue Scheduling} Strategie?

\tcblower

\justifying
\begin{center}
\includegraphics[max width=.9\textwidth]{betriebssysteme_skript_86.png}
\end{center}
\begin{center}
\includegraphics[max width=.9\textwidth]{betriebssysteme_skript_87.png}
\end{center}
\begin{center}
\includegraphics[max width=.9\textwidth]{betriebssysteme_skript_88.png}
\end{center}
\begin{center}
\includegraphics[max width=.9\textwidth]{betriebssysteme_skript_89.png}
\end{center}

\end{tcolorbox}
%---------------------
\begin{tcolorbox}[colback=white!10!white,colframe=lightgray!75!black,
  savelowerto=\jobname_ex.tex,breakable,enhanced,lines before break=40]

\justifying
Erläutern Sie die \textbf{MLFQ}-Strategie anhand von \textit{3 Queues} und dem \textit{Zeitscheibeverfahren}.<missing IDENTIFIER>

\tcblower

\justifying
\begin{center}
\includegraphics[max width=.9\textwidth]{betriebssysteme_skript_89.png}
\end{center}

\end{tcolorbox}
%---------------------
\begin{tcolorbox}[colback=white!10!white,colframe=lightgray!75!black,
  savelowerto=\jobname_ex.tex,breakable,enhanced,lines before break=40]

\justifying
Nennen Sie die \textit{wichtigsten} Parameter der \textbf{MLFQ}-Strategie.

\tcblower

\justifying
\begin{center}
\includegraphics[max width=.9\textwidth]{betriebssysteme_skript_90.png}
\end{center}

\end{tcolorbox}
%---------------------
\begin{tcolorbox}[colback=white!10!white,colframe=lightgray!75!black,
  savelowerto=\jobname_ex.tex,breakable,enhanced,lines before break=40]

\justifying
Beschreiben Sie die \textbf{Ausgangssituation} sowie die \textit{Kernidee} eines \textbf{CFS}. Wofür steht die \textit{Abkürzung?}

\tcblower

\justifying
\begin{center}
\includegraphics[max width=.9\textwidth]{betriebssysteme_skript_93.png}
\end{center}

\end{tcolorbox}
%---------------------
\begin{tcolorbox}[colback=white!10!white,colframe=lightgray!75!black,
  savelowerto=\jobname_ex.tex,breakable,enhanced,lines before break=40]

\justifying
Wie berechnet sich die \textit{virtuelle} Laufzeit eines \textbf{CFS} und wo liegt der Zusammenhang zur \textit{Priorität} eines Prozesses?

\tcblower

\justifying
\begin{center}
\includegraphics[max width=.9\textwidth]{betriebssysteme_skript_93.png}
\end{center}

\end{tcolorbox}
%---------------------
\begin{tcolorbox}[colback=white!10!white,colframe=lightgray!75!black,
  savelowerto=\jobname_ex.tex,breakable,enhanced,lines before break=40]

\justifying
Welche \textbf{Datenstruktur} verwendet ein \textbf{CFS} um seine Daten zu speichern?

\tcblower

\justifying
\begin{center}
\includegraphics[max width=.9\textwidth]{betriebssysteme_skript_94.png}
\end{center}

\end{tcolorbox}
%---------------------
\begin{tcolorbox}[colback=white!10!white,colframe=lightgray!75!black,
  savelowerto=\jobname_ex.tex,breakable,enhanced,lines before break=40]

\justifying
Erläutern Sie das Konzept einer \textbf{Task Group} im Zusammenhang mit dem \textbf{CFS}.<missing IDENTIFIER>

\tcblower

\justifying
\begin{center}
\includegraphics[max width=.9\textwidth]{betriebssysteme_skript_95.png}
\end{center}

\end{tcolorbox}
%---------------------
\begin{tcolorbox}[colback=white!10!white,colframe=lightgray!75!black,
  savelowerto=\jobname_ex.tex,breakable,enhanced,lines before break=40]

\justifying
Geben Sie einen \textit{kurzen} Überblick über den \textbf{BFS}.<missing IDENTIFIER>

\tcblower

\justifying
\begin{center}
\includegraphics[max width=.9\textwidth]{betriebssysteme_skript_97.png}
\end{center}

\end{tcolorbox}
%---------------------
\begin{tcolorbox}[colback=white!10!white,colframe=lightgray!75!black,
  savelowerto=\jobname_ex.tex,breakable,enhanced,lines before break=40]

\justifying
Beschreiben Sie grob was man unter dem sog. \textbf{Nebenläufigkeitsproblem} versteht. Erläutern Sie dies anhand eines Beispiels.

\tcblower

\justifying
\begin{center}
\includegraphics[max width=.9\textwidth]{betriebssysteme_skript_99.png}
\end{center}
\begin{center}
\includegraphics[max width=.9\textwidth]{betriebssysteme_skript_103.png}
\end{center}
\begin{center}
\includegraphics[max width=.9\textwidth]{betriebssysteme_skript_104.png}
\end{center}
\begin{center}
\includegraphics[max width=.9\textwidth]{betriebssysteme_skript_108.png}
\end{center}

\end{tcolorbox}
%---------------------
\begin{tcolorbox}[colback=white!10!white,colframe=lightgray!75!black,
  savelowerto=\jobname_ex.tex,breakable,enhanced,lines before break=40]

\justifying
Nennen Sie die beiden Möglichkeiten der Thread-Erzeugung und Linux. Worin liegt der Unterschied zur Erzeugung von Prozessen?

\tcblower

\justifying
\begin{center}
\includegraphics[max width=.9\textwidth]{betriebssysteme_skript_66.png}
\end{center}
\begin{center}
\includegraphics[max width=.9\textwidth]{betriebssysteme_skript_67.png}
\end{center}
\begin{center}
\includegraphics[max width=.9\textwidth]{betriebssysteme_skript_68.png}
\end{center}
\begin{center}
\includegraphics[max width=.9\textwidth]{betriebssysteme_skript_69.png}
\end{center}
\begin{center}
\includegraphics[max width=.9\textwidth]{betriebssysteme_skript_70.png}
\end{center}

\end{tcolorbox}
\Needspace{20\baselineskip}
\subsection{Interprozesskommunikation}
%---------------------
\begin{tcolorbox}[colback=white!10!white,colframe=lightgray!75!black,
  savelowerto=\jobname_ex.tex,breakable,enhanced,lines before break=40]

\justifying
Charakterisieren Sie die drei wesentlichen Probleme der \textbf{Nebenläufigkeit}.<missing IDENTIFIER>

\tcblower

\justifying
\begin{center}
\includegraphics[max width=.9\textwidth]{betriebssysteme_skript_112.png}
\end{center}
\begin{center}
\includegraphics[max width=.9\textwidth]{betriebssysteme_skript_113.png}
\end{center}
\begin{center}
\includegraphics[max width=.9\textwidth]{betriebssysteme_skript_114.png}
\end{center}
\begin{center}
\includegraphics[max width=.9\textwidth]{betriebssysteme_skript_115.png}
\end{center}

\end{tcolorbox}
%---------------------
\begin{tcolorbox}[colback=white!10!white,colframe=lightgray!75!black,
  savelowerto=\jobname_ex.tex,breakable,enhanced,lines before break=40]

\justifying
Erläutern Sie was man unter einem sog. \textit{kritischem Abschnitt} versteht. Nennen Sie den \textit{Fachbegriff} des Lösungsansatzes und beschreiben Sie die 4 \textbf{Bedingungen} für eine \textit{gute} Lösung.

\tcblower

\justifying
\begin{center}
\includegraphics[max width=.9\textwidth]{betriebssysteme_skript_118.png}
\end{center}

\end{tcolorbox}
%---------------------
\begin{tcolorbox}[colback=white!10!white,colframe=lightgray!75!black,
  savelowerto=\jobname_ex.tex,breakable,enhanced,lines before break=40]

\justifying
Erläutern Sie wieso es das \textbf{abschalten} von Interrupts zum Steueren eines \textit{kritischen Abschnitts} nicht als optimaler Lösungsansatz angesehen wird. Wo kann dieser Ansatz jedoch sinnvoll sein?

\tcblower

\justifying
\begin{center}
\includegraphics[max width=.9\textwidth]{betriebssysteme_skript_120.png}
\end{center}

\end{tcolorbox}
%---------------------
\begin{tcolorbox}[colback=white!10!white,colframe=lightgray!75!black,
  savelowerto=\jobname_ex.tex,breakable,enhanced,lines before break=40]

\justifying
Erläutern Sie die \textit{Strategie} des \textit{Sperrens von Variablen} zum regeln des Zugriffs auf einen kritischen Abschnitt.

\tcblower

\justifying
\begin{center}
\includegraphics[max width=.9\textwidth]{betriebssysteme_skript_120.png}
\end{center}
\begin{center}
\includegraphics[max width=.9\textwidth]{betriebssysteme_skript_121.png}
\end{center}
\begin{center}
\includegraphics[max width=.9\textwidth]{betriebssysteme_skript_122.png}
\end{center}
Das \textit{Kleingedruckte}: Scheduler könnte im falschen Moment unterbrechen.\begin{center}
\includegraphics[max width=.9\textwidth]{betriebssysteme_skript_122.png}
\end{center}

\end{tcolorbox}
%---------------------
\begin{tcolorbox}[colback=white!10!white,colframe=lightgray!75!black,
  savelowerto=\jobname_ex.tex,breakable,enhanced,lines before break=40]

\justifying
Beschreiben Sie die \textit{Lösung von Peterson} zum regeln des Zugriffs auf einen kritischen Abschnitt. Auf wie viele \textit{Threads} kann dieser Ansatz angewendet werden? Beschreiben Sie neben den Vorteilen auch die Nachteile dieses Ansatzes.

\tcblower

\justifying
\begin{center}
\includegraphics[max width=.9\textwidth]{betriebssysteme_skript_126.png}
\end{center}
\begin{center}
\includegraphics[max width=.9\textwidth]{betriebssysteme_skript_127.png}
\end{center}
\begin{center}
\includegraphics[max width=.9\textwidth]{betriebssysteme_skript_128.png}
\end{center}
\begin{center}
\includegraphics[max width=.9\textwidth]{betriebssysteme_skript_129.png}
\end{center}
\begin{center}
\includegraphics[max width=.9\textwidth]{betriebssysteme_skript_130.png}
\end{center}

\end{tcolorbox}
%---------------------
\begin{tcolorbox}[colback=white!10!white,colframe=lightgray!75!black,
  savelowerto=\jobname_ex.tex,breakable,enhanced,lines before break=40]

\justifying
Was versteht man im Zusammenhang mit der Verarbeitung eines \textbf{kritischen Abschnitts} unter \textit{TSL-Operationen?}

\tcblower

\justifying
\begin{center}
\includegraphics[max width=.9\textwidth]{betriebssysteme_skript_132.png}
\end{center}
\begin{center}
\includegraphics[max width=.9\textwidth]{betriebssysteme_skript_133.png}
\end{center}
\begin{center}
\includegraphics[max width=.9\textwidth]{betriebssysteme_skript_134.png}
\end{center}

\end{tcolorbox}
%---------------------
\begin{tcolorbox}[colback=white!10!white,colframe=lightgray!75!black,
  savelowerto=\jobname_ex.tex,breakable,enhanced,lines before break=40]

\justifying
Erläutern Sie die Lösungsmöglichkeit für den \textit{wechselseitigen Ausschluss} mittels einer \textbf{Semaphore}. Welche \textit{Operationen} stellt dieses Verfahren zur Verfügung?

\tcblower

\justifying
\begin{center}
\includegraphics[max width=.9\textwidth]{betriebssysteme_skript_136.png}
\end{center}
\begin{center}
\includegraphics[max width=.9\textwidth]{betriebssysteme_skript_137.png}
\end{center}
\begin{center}
\includegraphics[max width=.9\textwidth]{betriebssysteme_skript_138.png}
\end{center}
\begin{center}
\includegraphics[max width=.9\textwidth]{betriebssysteme_skript_140.png}
\end{center}

\end{tcolorbox}
%---------------------
\begin{tcolorbox}[colback=white!10!white,colframe=lightgray!75!black,
  savelowerto=\jobname_ex.tex,breakable,enhanced,lines before break=40]

\justifying
Erläutern Sie die \textit{verschiedenen} Möglichkeiten zur Realisierung von \textbf{atomaren} Operationen.Beschreiben Sie in diesem Zusammenhang noch kurz den \textit{Unterschied} zwischen der sog. \textit{Quasi-Paralleltiät} sowie der \textit{echten Paralleltität}.<missing IDENTIFIER>

\tcblower

\justifying
\begin{center}
\includegraphics[max width=.9\textwidth]{betriebssysteme_skript_141.png}
\end{center}

\end{tcolorbox}
%---------------------
\begin{tcolorbox}[colback=white!10!white,colframe=lightgray!75!black,
  savelowerto=\jobname_ex.tex,breakable,enhanced,lines before break=40]

\justifying
Beschreiben Sie den zeitlichen Ablauf des \textit{wechselseiten Ausschlusses} unter verwendung einer \textbf{Semaphore} anhand eines Beispiels.

\tcblower

\justifying
\begin{center}
\includegraphics[max width=.9\textwidth]{betriebssysteme_skript_136.png}
\end{center}
\begin{center}
\includegraphics[max width=.9\textwidth]{betriebssysteme_skript_142.png}
\end{center}
\begin{center}
\includegraphics[max width=.9\textwidth]{betriebssysteme_skript_145.png}
\end{center}
\begin{center}
\includegraphics[max width=.9\textwidth]{betriebssysteme_skript_146.png}
\end{center}
\begin{center}
\includegraphics[max width=.9\textwidth]{betriebssysteme_skript_147.png}
\end{center}

\end{tcolorbox}
%---------------------
\begin{tcolorbox}[colback=white!10!white,colframe=lightgray!75!black,
  savelowerto=\jobname_ex.tex,breakable,enhanced,lines before break=40]

\justifying
Beschreiben Sie das Konzept des sog. \textbf{Monitors}. Wofür ist es geeignet und welches bekannte Verfahren soll es ersetzen?Gehen Sie auch auf die Nachteile von Monitoren ein.

\tcblower

\justifying
\begin{center}
\includegraphics[max width=.9\textwidth]{betriebssysteme_skript_149.png}
\end{center}
\begin{center}
\includegraphics[max width=.9\textwidth]{betriebssysteme_skript_150.png}
\end{center}

\end{tcolorbox}
\Needspace{20\baselineskip}
\subsection{Lernkontrolle}
%---------------------
\begin{tcolorbox}[colback=white!10!white,colframe=lightgray!75!black,
  savelowerto=\jobname_ex.tex,breakable,enhanced,lines before break=40]

\justifying
Richtig oder Falsch: Die Abfolge von \textit{Programmabläufen} unterschiedlicher Prozesse kann immer durch deren Reihenfolge im Source-Code \textit{reproduzierbar} festgelegt werden?

\tcblower

\justifying
Falsch: Durch die Zuordnung der Prozesse zur CPU und Prozesse mit höherer Priorität sind zeitliche Programmabläufe selten repduzierbar.
\end{tcolorbox}
%---------------------
\begin{tcolorbox}[colback=white!10!white,colframe=lightgray!75!black,
  savelowerto=\jobname_ex.tex,breakable,enhanced,lines before break=40]

\justifying
Beschreiben Sie die Aufgaben eines \textbf{Prozesskontrollblocks} (PCB) und geben sie an \textit{wo} dieser abgespeichert wird.

\tcblower

\justifying
PCB ist eine Datenstruktur, die genutzt wird, um assoziierte dAten für einen Prozess zu speichern.PCBs werden in der Prozesstabelle des BS gespeichert.
\end{tcolorbox}
%---------------------
\begin{tcolorbox}[colback=white!10!white,colframe=lightgray!75!black,
  savelowerto=\jobname_ex.tex,breakable,enhanced,lines before break=40]

\justifying
Nennen und beschreiben Sie die drei Zustände eines Prozesses gemäss dem Standardschema für das Zustandsmodell.

\tcblower

\justifying
- Bereit- Rechnend- Blockiert
\end{tcolorbox}
%---------------------
\begin{tcolorbox}[colback=white!10!white,colframe=lightgray!75!black,
  savelowerto=\jobname_ex.tex,breakable,enhanced,lines before break=40]

\justifying
Richtig oder Falsch: Ein Thread hat einen eigenen Stack und eingene CPU Register, Threads eines Prozesses teilen sich aber den gleichen Speicher innerhalb dieses Prozesses.

\tcblower

\justifying
Richtig!
\end{tcolorbox}
%---------------------
\begin{tcolorbox}[colback=white!10!white,colframe=lightgray!75!black,
  savelowerto=\jobname_ex.tex,breakable,enhanced,lines before break=40]

\justifying
Was wird mit dem UNIX-Befehl \textit{fork} erzeugt?

\tcblower

\justifying
Eine exakte Kopie des Elternprozesses.
\end{tcolorbox}
%*********************
\Needspace{20\baselineskip}
\section{Generelles}
%---------------------
\begin{tcolorbox}[colback=white!10!white,colframe=lightgray!75!black,
  savelowerto=\jobname_ex.tex,breakable,enhanced,lines before break=40]

\justifying
Betriebssysteme::Lernkontrolle

\tcblower

\justifying

\end{tcolorbox}

\end{document}


