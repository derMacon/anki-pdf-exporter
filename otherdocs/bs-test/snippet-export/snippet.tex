\documentclass{article}
\usepackage{tikz,lipsum,lmodern}
\usepackage[most]{tcolorbox}
\usepackage[paperheight=10.75in,paperwidth=7.25in,margin=1in,heightrounded]{geometry}
\usepackage{graphicx}
\usepackage{blindtext}
\usepackage{ragged2e}
\usepackage[space]{grffile}
\usepackage[utf8]{inputenc}

\graphicspath{{"/home/silasUser/.local/share/Anki2/User 1/collection.media/"}}

\title{todo extract deck name}

\begin{document}
%*********************
\section{Generelles}
%---------------------
\begin{tcolorbox}[colback=white!10!white,colframe=lightgray!75!black,
  savelowerto=\jobname_ex.tex]

\begin{center}
Charakterisieren Sie das BIOS und nennen Sie die wichtigsten Aufgaben. Wofuer steht die Abkuerzung ueberhaupt?

\end{center}

\tcblower

\justifying
\includegraphics[width=.9\textwidth]{betriebssysteme_skript_12.png}

\end{tcolorbox}
%---------------------
\begin{tcolorbox}[colback=white!10!white,colframe=lightgray!75!black,
  savelowerto=\jobname_ex.tex]

\begin{center}
::Einführung

\end{center}

\tcblower

\justifying

\end{tcolorbox}
%---------------------
\begin{tcolorbox}[colback=white!10!white,colframe=lightgray!75!black,
  savelowerto=\jobname_ex.tex]

\begin{center}
Richtig oder falsch: Ältere Betriebssysteme wie MS-DOS benutzten Basisfunktionen des BIOS auch nach dem Laden des eigentlichen BS.

\end{center}

\tcblower

\justifying
Richtig. Neuere Varianten tun dies nicht mehr, sodass man gar den Chip des BIOS während des laufenden Betriebs entfernen kann.
\includegraphics[width=.9\textwidth]{paste-17304423235585.png}

\end{tcolorbox}
%---------------------
\begin{tcolorbox}[colback=white!10!white,colframe=lightgray!75!black,
  savelowerto=\jobname_ex.tex]

\begin{center}
::Einführung

\end{center}

\tcblower

\justifying

\end{tcolorbox}
%---------------------
\begin{tcolorbox}[colback=white!10!white,colframe=lightgray!75!black,
  savelowerto=\jobname_ex.tex]

\begin{center}
Erläutern Sie den Begriff des 
\textbf{UEFI
}wofür steht diese Abkürzung, woher kommt es und welche Vorteile bietet es im Gegensatz zum BIOS?
Erläutern Sie ebenfalls kurz was man unter dem Begriff 
\textit{Firmware 
}versteht.
Welche Nachteile bietet das Booten mittels UEFI?

\end{center}

\tcblower

\justifying
\includegraphics[width=.9\textwidth]{paste-17836999180289 (1).png}

\end{tcolorbox}
%---------------------
\begin{tcolorbox}[colback=white!10!white,colframe=lightgray!75!black,
  savelowerto=\jobname_ex.tex]

\begin{center}
::Einführung

\end{center}

\tcblower

\justifying

\end{tcolorbox}
%---------------------
\begin{tcolorbox}[colback=white!10!white,colframe=lightgray!75!black,
  savelowerto=\jobname_ex.tex]

\begin{center}
Skizzieren Sie den 
\textbf{Bootvorgang
} mittels UEFI und grenzen Sie diesen von dem mittels des klassischen BIOS ab.

\end{center}

\tcblower

\justifying
\includegraphics[width=.9\textwidth]{paste-21775484190721.png}

\end{tcolorbox}
%---------------------
\begin{tcolorbox}[colback=white!10!white,colframe=lightgray!75!black,
  savelowerto=\jobname_ex.tex]

\begin{center}
::Einführung

\end{center}

\tcblower

\justifying

\end{tcolorbox}
%---------------------
\begin{tcolorbox}[colback=white!10!white,colframe=lightgray!75!black,
  savelowerto=\jobname_ex.tex]

\begin{center}
Erläutern Sie die BS
\textit{Dienstleistung
}der 
\textbf{Abstraktion
}von konkreter Hardware. Wieso ist so etwas sinnvoll und gibt es hierbei Ausnahmen?

\end{center}

\tcblower

\justifying
\includegraphics[width=.9\textwidth]{paste-22213570854913.png}

\end{tcolorbox}
%---------------------
\begin{tcolorbox}[colback=white!10!white,colframe=lightgray!75!black,
  savelowerto=\jobname_ex.tex]

\begin{center}
::Einführung

\end{center}

\tcblower

\justifying

\end{tcolorbox}
%---------------------
\begin{tcolorbox}[colback=white!10!white,colframe=lightgray!75!black,
  savelowerto=\jobname_ex.tex]

\begin{center}
Welche Dienstleistungen stellt ein Betriebssystem im Bereich der 
\textbf{Speichermedien
} bereit?

\end{center}

\tcblower

\justifying
\includegraphics[width=.9\textwidth]{paste-22617297780737.png}

\end{tcolorbox}
%---------------------
\begin{tcolorbox}[colback=white!10!white,colframe=lightgray!75!black,
  savelowerto=\jobname_ex.tex]

\begin{center}
::Einführung

\end{center}

\tcblower

\justifying

\end{tcolorbox}
%---------------------
\begin{tcolorbox}[colback=white!10!white,colframe=lightgray!75!black,
  savelowerto=\jobname_ex.tex]

\begin{center}
Welche Dienstleistungen stellt ein BS im Bereich der 
\textbf{Prozessverwaltung
}bereit? Erläutern Sie ebenfalls diesen Begriff.

\end{center}

\tcblower

\justifying
\includegraphics[width=.9\textwidth]{paste-22965190131713.png}

\end{tcolorbox}
%---------------------
\begin{tcolorbox}[colback=white!10!white,colframe=lightgray!75!black,
  savelowerto=\jobname_ex.tex]

\begin{center}
::Einführung

\end{center}

\tcblower

\justifying

\end{tcolorbox}
%---------------------
\begin{tcolorbox}[colback=white!10!white,colframe=lightgray!75!black,
  savelowerto=\jobname_ex.tex]

\begin{center}
Nennen und beschreiben Sie alle wesentlichen Dienstleistungen die ein Betriebssystem zur Verfügung stellt.

\end{center}

\tcblower

\justifying
\includegraphics[width=.9\textwidth]{paste-23287312678913.png}
\includegraphics[width=.9\textwidth]{paste-23300197580801.png}
\includegraphics[width=.9\textwidth]{paste-23313082482689.png}
\includegraphics[width=.9\textwidth]{paste-22960895164417.png}

\end{tcolorbox}
%---------------------
\begin{tcolorbox}[colback=white!10!white,colframe=lightgray!75!black,
  savelowerto=\jobname_ex.tex]

\begin{center}
::Einführung marked

\end{center}

\tcblower

\justifying

\end{tcolorbox}
%---------------------
\begin{tcolorbox}[colback=white!10!white,colframe=lightgray!75!black,
  savelowerto=\jobname_ex.tex]

\begin{center}
Beschreiben Sie grob die Marktanteile der PC-Betriebssysteme.

\end{center}

\tcblower

\justifying
\includegraphics[width=.9\textwidth]{paste-23476291239937.png}
\includegraphics[width=.9\textwidth]{paste-23489176141825 (1).png}
\includegraphics[width=.9\textwidth]{paste-23510650978305 (1).png}

\end{tcolorbox}
%---------------------
\begin{tcolorbox}[colback=white!10!white,colframe=lightgray!75!black,
  savelowerto=\jobname_ex.tex]

\begin{center}
::Einführung

\end{center}

\tcblower

\justifying

\end{tcolorbox}
%---------------------
\begin{tcolorbox}[colback=white!10!white,colframe=lightgray!75!black,
  savelowerto=\jobname_ex.tex]

\begin{center}
Beschreiben Sie grob die Marktanteile der 
\textbf{mobilen-Betriebssysteme
}.

\end{center}

\tcblower

\justifying
\includegraphics[width=.9\textwidth]{paste-23708219473921.png}
\includegraphics[width=.9\textwidth]{paste-23721104375809.png}

\end{tcolorbox}
%---------------------
\begin{tcolorbox}[colback=white!10!white,colframe=lightgray!75!black,
  savelowerto=\jobname_ex.tex]

\begin{center}
::Einführung

\end{center}

\tcblower

\justifying

\end{tcolorbox}
%---------------------
\begin{tcolorbox}[colback=white!10!white,colframe=lightgray!75!black,
  savelowerto=\jobname_ex.tex]

\begin{center}
Nennen Sie die Besonderheiten von 
\textbf{Server-Betriebssystemen
}.

\end{center}

\tcblower

\justifying
\includegraphics[width=.9\textwidth]{paste-23794118819841.png}

\end{tcolorbox}
%---------------------
\begin{tcolorbox}[colback=white!10!white,colframe=lightgray!75!black,
  savelowerto=\jobname_ex.tex]

\begin{center}
::Einführung

\end{center}

\tcblower

\justifying

\end{tcolorbox}
%---------------------
\begin{tcolorbox}[colback=white!10!white,colframe=lightgray!75!black,
  savelowerto=\jobname_ex.tex]

\begin{center}
Nennen Sie die Besonderheiten von 
\textbf{Echtzeitbetriebssystemen
}. Und nennen Sie einige Beispiele.

\end{center}

\tcblower

\justifying
\includegraphics[width=.9\textwidth]{paste-23935852740609.png}
\includegraphics[width=.9\textwidth]{paste-23948737642497.png}

\end{tcolorbox}
%---------------------
\begin{tcolorbox}[colback=white!10!white,colframe=lightgray!75!black,
  savelowerto=\jobname_ex.tex]

\begin{center}
::Einführung

\end{center}

\tcblower

\justifying

\end{tcolorbox}
%---------------------
\begin{tcolorbox}[colback=white!10!white,colframe=lightgray!75!black,
  savelowerto=\jobname_ex.tex]

\begin{center}
Fassen Sie die verschiedenen Arten von Betriebssystemen zusammen.

\end{center}

\tcblower

\justifying
\includegraphics[width=.9\textwidth]{paste-24163486007297.png}
\includegraphics[width=.9\textwidth]{paste-24146306138113.png}

\end{tcolorbox}
%---------------------
\begin{tcolorbox}[colback=white!10!white,colframe=lightgray!75!black,
  savelowerto=\jobname_ex.tex]

\begin{center}
::Einführung

\end{center}

\tcblower

\justifying

\end{tcolorbox}
%---------------------
\begin{tcolorbox}[colback=white!10!white,colframe=lightgray!75!black,
  savelowerto=\jobname_ex.tex]

\begin{center}
Was ist ein 
\textbf{Kernel
}und woraus bestehen seine wesentlichen Aufgaben?

\end{center}

\tcblower

\justifying
\textbf{Zusammenfassung - Kernel:
}Software Kern des Betriebssystems, ist selbst ein Programm welches Dienste bereitstellt.
\includegraphics[width=.9\textwidth]{paste-24313809862657.png}

\end{tcolorbox}
%---------------------
\begin{tcolorbox}[colback=white!10!white,colframe=lightgray!75!black,
  savelowerto=\jobname_ex.tex]

\begin{center}
::Einführung

\end{center}

\tcblower

\justifying

\end{tcolorbox}
%---------------------
\begin{tcolorbox}[colback=white!10!white,colframe=lightgray!75!black,
  savelowerto=\jobname_ex.tex]

\begin{center}
Nennen Sie die Hauptarten von BS-Kernel.

\end{center}

\tcblower

\justifying
\includegraphics[width=.9\textwidth]{paste-24902220382209 (1).png}

\end{tcolorbox}
%---------------------
\begin{tcolorbox}[colback=white!10!white,colframe=lightgray!75!black,
  savelowerto=\jobname_ex.tex]

\begin{center}
::Einführung

\end{center}

\tcblower

\justifying

\end{tcolorbox}
%---------------------
\begin{tcolorbox}[colback=white!10!white,colframe=lightgray!75!black,
  savelowerto=\jobname_ex.tex]

\begin{center}
Beschreiben Sie die 
\textit{Philosophie, Nachteile
}und die 
\textit{Vorteile
}eines 
\textbf{monolithischen Kernel BS
}.

\end{center}

\tcblower

\justifying
\includegraphics[width=.9\textwidth]{paste-25082609008641 (1).png}

\end{tcolorbox}
%---------------------
\begin{tcolorbox}[colback=white!10!white,colframe=lightgray!75!black,
  savelowerto=\jobname_ex.tex]

\begin{center}
::Einführung

\end{center}

\tcblower

\justifying

\end{tcolorbox}
%---------------------
\begin{tcolorbox}[colback=white!10!white,colframe=lightgray!75!black,
  savelowerto=\jobname_ex.tex]

\begin{center}
Beschreiben Sie die
\textit{Philosophie, Nachteile
}und die
\textit{Vorteile
}eines
\textbf{Mikrokernel BS
}.

\end{center}

\tcblower

\justifying
\includegraphics[width=.9\textwidth]{paste-25232932864001.png}

\end{tcolorbox}
%---------------------
\begin{tcolorbox}[colback=white!10!white,colframe=lightgray!75!black,
  savelowerto=\jobname_ex.tex]

\begin{center}
::Einführung

\end{center}

\tcblower

\justifying

\end{tcolorbox}
%---------------------
\begin{tcolorbox}[colback=white!10!white,colframe=lightgray!75!black,
  savelowerto=\jobname_ex.tex]

\begin{center}
Beschreiben Sie die
\textit{Philosophie, Nachteile
}und die
\textit{Vorteile
}eines 
\textbf{hybridenBS Kernels
}.

\end{center}

\tcblower

\justifying
\includegraphics[width=.9\textwidth]{paste-25701084299265 (1).png}

\end{tcolorbox}
%---------------------
\begin{tcolorbox}[colback=white!10!white,colframe=lightgray!75!black,
  savelowerto=\jobname_ex.tex]

\begin{center}
::Einführung

\end{center}

\tcblower

\justifying

\end{tcolorbox}
%---------------------
\begin{tcolorbox}[colback=white!10!white,colframe=lightgray!75!black,
  savelowerto=\jobname_ex.tex]

\begin{center}
Fassen Sie noch einmal die verschiedenen Klassen von Kernels zusammen und grenzen Sie diese voneinander ab.

\end{center}

\tcblower

\justifying
\includegraphics[width=.9\textwidth]{paste-25756918874113.png}
\includegraphics[width=.9\textwidth]{paste-25769803776001.png}
\includegraphics[width=.9\textwidth]{paste-25786983645185.png}
\includegraphics[width=.9\textwidth]{paste-25799868547073.png}

\end{tcolorbox}
%---------------------
\begin{tcolorbox}[colback=white!10!white,colframe=lightgray!75!black,
  savelowerto=\jobname_ex.tex]

\begin{center}
::Einführung

\end{center}

\tcblower

\justifying

\end{tcolorbox}
%---------------------
\begin{tcolorbox}[colback=white!10!white,colframe=lightgray!75!black,
  savelowerto=\jobname_ex.tex]

\begin{center}
Nennen Sie die wesentlichen Vorteile von 
\textbf{UEFI
} gegenüber 
\textbf{BIOS
}!<missing IDENTIFIER>

\end{center}

\tcblower

\justifying
\includegraphics[width=.9\textwidth]{betriebssysteme_skript_49.png}

\end{tcolorbox}
%---------------------
\begin{tcolorbox}[colback=white!10!white,colframe=lightgray!75!black,
  savelowerto=\jobname_ex.tex]

\begin{center}
::Einführung

\end{center}

\tcblower

\justifying

\end{tcolorbox}
%---------------------
\begin{tcolorbox}[colback=white!10!white,colframe=lightgray!75!black,
  savelowerto=\jobname_ex.tex]

\begin{center}
Betriebssysteme::Lernkontrolle

\end{center}

\tcblower

\justifying

\end{tcolorbox}
%---------------------
\begin{tcolorbox}[colback=white!10!white,colframe=lightgray!75!black,
  savelowerto=\jobname_ex.tex]

\begin{center}
Nennen Sie die wesentlichen Dienstleistungen eines Betriebssystems

\end{center}

\tcblower

\justifying
\includegraphics[width=.9\textwidth]{betriebssysteme_skript_49.png}

\end{tcolorbox}
%---------------------
\begin{tcolorbox}[colback=white!10!white,colframe=lightgray!75!black,
  savelowerto=\jobname_ex.tex]

\begin{center}
::Einführung Betriebssysteme::Lernkontrolle

\end{center}

\tcblower

\justifying

\end{tcolorbox}
%---------------------
\begin{tcolorbox}[colback=white!10!white,colframe=lightgray!75!black,
  savelowerto=\jobname_ex.tex]

\begin{center}
Welche der Kernel Aufgaben gehört 
\textbf{NICHT
} zu den Aufgaben eines BS-Kerns?

\end{center}

\tcblower

\justifying
\includegraphics[width=.9\textwidth]{betriebssysteme_skript_50.png}

\end{tcolorbox}
%---------------------
\begin{tcolorbox}[colback=white!10!white,colframe=lightgray!75!black,
  savelowerto=\jobname_ex.tex]

\begin{center}
::Einführung Betriebssysteme::Lernkontrolle marked

\end{center}

\tcblower

\justifying

\end{tcolorbox}
%---------------------
\begin{tcolorbox}[colback=white!10!white,colframe=lightgray!75!black,
  savelowerto=\jobname_ex.tex]

\begin{center}
Beschreiben Sie die drei Hauptarten von BS-Kernel und gehen Sie dabei auf die Vor- und Nachteile ein!

\end{center}

\tcblower

\justifying
\includegraphics[width=.9\textwidth]{betriebssysteme_skript_50.png}

\end{tcolorbox}
%---------------------
\begin{tcolorbox}[colback=white!10!white,colframe=lightgray!75!black,
  savelowerto=\jobname_ex.tex]

\begin{center}
::Einführung Betriebssysteme::Lernkontrolle

\end{center}

\tcblower

\justifying

\end{tcolorbox}
%---------------------
\begin{tcolorbox}[colback=white!10!white,colframe=lightgray!75!black,
  savelowerto=\jobname_ex.tex]

\begin{center}
Definieren Sie den Begriff des 
\textbf{Prozesses
}. 
Was versteht man in diesem Zusammenhang unter der der 
\textit{Quasiparallelität
}?<missing IDENTIFIER>

\end{center}

\tcblower

\justifying
\includegraphics[width=.9\textwidth]{betriebssysteme_skript_54.png}
\includegraphics[width=.9\textwidth]{betriebssysteme_skript_54.png}

\end{tcolorbox}
%---------------------
\begin{tcolorbox}[colback=white!10!white,colframe=lightgray!75!black,
  savelowerto=\jobname_ex.tex]

\begin{center}
::ProzesseUndThreads

\end{center}

\tcblower

\justifying

\end{tcolorbox}
%---------------------
\begin{tcolorbox}[colback=white!10!white,colframe=lightgray!75!black,
  savelowerto=\jobname_ex.tex]

\begin{center}
Was versteht man unter dem sog. 
\textbf{Prozessmodell
}?<missing IDENTIFIER>
Erläutern Sie in diesem Zusammenhang grob, was man unter einer 
\textit{Schedulingstrategie
} versteht.

\end{center}

\tcblower

\justifying
\includegraphics[width=.9\textwidth]{betriebssysteme_skript_54.png}
\includegraphics[width=.9\textwidth]{betriebssysteme_skript_54.png}

\end{tcolorbox}
%---------------------
\begin{tcolorbox}[colback=white!10!white,colframe=lightgray!75!black,
  savelowerto=\jobname_ex.tex]

\begin{center}
::ProzesseUndThreads

\end{center}

\tcblower

\justifying

\end{tcolorbox}
%---------------------
\begin{tcolorbox}[colback=white!10!white,colframe=lightgray!75!black,
  savelowerto=\jobname_ex.tex]

\begin{center}
Beschreiben Sie den Begriff der 
\textbf{Prozesstabelle
}.
Welche Daten werden hierbei festgehalten?

\end{center}

\tcblower

\justifying
\includegraphics[width=.9\textwidth]{betriebssysteme_skript_55.png}

\end{tcolorbox}
%---------------------
\begin{tcolorbox}[colback=white!10!white,colframe=lightgray!75!black,
  savelowerto=\jobname_ex.tex]

\begin{center}
::ProzesseUndThreads marked

\end{center}

\tcblower

\justifying

\end{tcolorbox}
%---------------------
\begin{tcolorbox}[colback=white!10!white,colframe=lightgray!75!black,
  savelowerto=\jobname_ex.tex]

\begin{center}
Was versteht man unter einem 
\textit{Prozesskontrollblock
}?<missing IDENTIFIER>

\end{center}

\tcblower

\justifying
\includegraphics[width=.9\textwidth]{betriebssysteme_skript_55.png}
\includegraphics[width=.9\textwidth]{betriebssysteme_skript_56.png}

\end{tcolorbox}
%---------------------
\begin{tcolorbox}[colback=white!10!white,colframe=lightgray!75!black,
  savelowerto=\jobname_ex.tex]

\begin{center}
::ProzesseUndThreads

\end{center}

\tcblower

\justifying

\end{tcolorbox}
%---------------------
\begin{tcolorbox}[colback=white!10!white,colframe=lightgray!75!black,
  savelowerto=\jobname_ex.tex]

\begin{center}
Erläutern Sie den Begriff der 
\textbf{Prozesserzeugung
}. 
Wann ist so etwas sinnvoll, welche Möglichkeiten der Erzeugung gibt es, welche Befehle werden verwendet?

\end{center}

\tcblower

\justifying
\includegraphics[width=.9\textwidth]{betriebssysteme_skript_57.png}

\end{tcolorbox}
%---------------------
\begin{tcolorbox}[colback=white!10!white,colframe=lightgray!75!black,
  savelowerto=\jobname_ex.tex]

\begin{center}
::ProzesseUndThreads

\end{center}

\tcblower

\justifying

\end{tcolorbox}
%---------------------
\begin{tcolorbox}[colback=white!10!white,colframe=lightgray!75!black,
  savelowerto=\jobname_ex.tex]

\begin{center}
Wie kann ein Prozess beendet werden?
Welche unterschiedlichen Arten gibt es, erläutern Sie dies am Beispiel vom BS Linux.

\end{center}

\tcblower

\justifying
\includegraphics[width=.9\textwidth]{betriebssysteme_skript_58.png}

\end{tcolorbox}
%---------------------
\begin{tcolorbox}[colback=white!10!white,colframe=lightgray!75!black,
  savelowerto=\jobname_ex.tex]

\begin{center}
::ProzesseUndThreads

\end{center}

\tcblower

\justifying

\end{tcolorbox}

\end{document}


