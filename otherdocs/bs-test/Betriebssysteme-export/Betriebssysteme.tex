\documentclass{article}
\usepackage{tikz,lipsum,lmodern}
\usepackage[most]{tcolorbox}
\usepackage[paperheight=10.75in,paperwidth=7.25in,margin=1in,heightrounded]{geometry}
\usepackage{graphicx}
\usepackage{blindtext}
\usepackage{ragged2e}
\usepackage[space]{grffile}
\usepackage[utf8]{inputenc}

\graphicspath{{"/home/silasUser/.local/share/Anki2/User 1/collection.media/"}}

\title{todo extract deck name}

\begin{document}
%*********************
\section{Generelles}
%---------------------
\begin{tcolorbox}[colback=white!10!white,colframe=lightgray!75!black,
  savelowerto=\jobname_ex.tex]

\begin{center}
Charakterisieren Sie das BIOS und nennen Sie die wichtigsten Aufgaben. Wofuer steht die Abkuerzung ueberhaupt?

\end{center}

\tcblower

\justifying
\includegraphics[width=.9\textwidth]{betriebssysteme_skript_12.png}

\end{tcolorbox}
%---------------------
\begin{tcolorbox}[colback=white!10!white,colframe=lightgray!75!black,
  savelowerto=\jobname_ex.tex]

\begin{center}
Richtig oder falsch: Ältere Betriebssysteme wie MS-DOS benutzten Basisfunktionen des BIOS auch nach dem Laden des eigentlichen BS.

\end{center}

\tcblower

\justifying
Richtig. Neuere Varianten tun dies nicht mehr, sodass man gar den Chip des BIOS während des laufenden Betriebs entfernen kann.
\includegraphics[width=.9\textwidth]{paste-17304423235585.png}

\end{tcolorbox}
%---------------------
\begin{tcolorbox}[colback=white!10!white,colframe=lightgray!75!black,
  savelowerto=\jobname_ex.tex]

\begin{center}
Erläutern Sie den Begriff des 
\textbf{UEFI
}wofür steht diese Abkürzung, woher kommt es und welche Vorteile bietet es im Gegensatz zum BIOS?
Erläutern Sie ebenfalls kurz was man unter dem Begriff 
\textit{Firmware 
}versteht.
Welche Nachteile bietet das Booten mittels UEFI?

\end{center}

\tcblower

\justifying
\includegraphics[width=.9\textwidth]{paste-17836999180289 (1).png}

\end{tcolorbox}
%---------------------
\begin{tcolorbox}[colback=white!10!white,colframe=lightgray!75!black,
  savelowerto=\jobname_ex.tex]

\begin{center}
Skizzieren Sie den 
\textbf{Bootvorgang
} mittels UEFI und grenzen Sie diesen von dem mittels des klassischen BIOS ab.

\end{center}

\tcblower

\justifying
\includegraphics[width=.9\textwidth]{paste-21775484190721.png}

\end{tcolorbox}
%---------------------
\begin{tcolorbox}[colback=white!10!white,colframe=lightgray!75!black,
  savelowerto=\jobname_ex.tex]

\begin{center}
Erläutern Sie die BS
\textit{Dienstleistung
}der 
\textbf{Abstraktion
}von konkreter Hardware. Wieso ist so etwas sinnvoll und gibt es hierbei Ausnahmen?

\end{center}

\tcblower

\justifying
\includegraphics[width=.9\textwidth]{paste-22213570854913.png}

\end{tcolorbox}
%---------------------
\begin{tcolorbox}[colback=white!10!white,colframe=lightgray!75!black,
  savelowerto=\jobname_ex.tex]

\begin{center}
Welche Dienstleistungen stellt ein Betriebssystem im Bereich der 
\textbf{Speichermedien
} bereit?

\end{center}

\tcblower

\justifying
\includegraphics[width=.9\textwidth]{paste-22617297780737.png}

\end{tcolorbox}
%---------------------
\begin{tcolorbox}[colback=white!10!white,colframe=lightgray!75!black,
  savelowerto=\jobname_ex.tex]

\begin{center}
Welche Dienstleistungen stellt ein BS im Bereich der 
\textbf{Prozessverwaltung
}bereit? Erläutern Sie ebenfalls diesen Begriff.

\end{center}

\tcblower

\justifying
\includegraphics[width=.9\textwidth]{paste-22965190131713.png}

\end{tcolorbox}
%---------------------
\begin{tcolorbox}[colback=white!10!white,colframe=lightgray!75!black,
  savelowerto=\jobname_ex.tex]

\begin{center}
Nennen und beschreiben Sie alle wesentlichen Dienstleistungen die ein Betriebssystem zur Verfügung stellt.

\end{center}

\tcblower

\justifying
\includegraphics[width=.9\textwidth]{paste-23287312678913.png}
\includegraphics[width=.9\textwidth]{paste-23300197580801.png}
\includegraphics[width=.9\textwidth]{paste-23313082482689.png}
\includegraphics[width=.9\textwidth]{paste-22960895164417.png}

\end{tcolorbox}
%---------------------
\begin{tcolorbox}[colback=white!10!white,colframe=lightgray!75!black,
  savelowerto=\jobname_ex.tex]

\begin{center}
Beschreiben Sie grob die Marktanteile der PC-Betriebssysteme.

\end{center}

\tcblower

\justifying
\includegraphics[width=.9\textwidth]{paste-23476291239937.png}
\includegraphics[width=.9\textwidth]{paste-23489176141825 (1).png}
\includegraphics[width=.9\textwidth]{paste-23510650978305 (1).png}

\end{tcolorbox}
%---------------------
\begin{tcolorbox}[colback=white!10!white,colframe=lightgray!75!black,
  savelowerto=\jobname_ex.tex]

\begin{center}
Beschreiben Sie grob die Marktanteile der 
\textbf{mobilen-Betriebssysteme
}.

\end{center}

\tcblower

\justifying
\includegraphics[width=.9\textwidth]{paste-23708219473921.png}
\includegraphics[width=.9\textwidth]{paste-23721104375809.png}

\end{tcolorbox}
%---------------------
\begin{tcolorbox}[colback=white!10!white,colframe=lightgray!75!black,
  savelowerto=\jobname_ex.tex]

\begin{center}
Nennen Sie die Besonderheiten von 
\textbf{Server-Betriebssystemen
}.

\end{center}

\tcblower

\justifying
\includegraphics[width=.9\textwidth]{paste-23794118819841.png}

\end{tcolorbox}
%---------------------
\begin{tcolorbox}[colback=white!10!white,colframe=lightgray!75!black,
  savelowerto=\jobname_ex.tex]

\begin{center}
Nennen Sie die Besonderheiten von 
\textbf{Echtzeitbetriebssystemen
}. Und nennen Sie einige Beispiele.

\end{center}

\tcblower

\justifying
\includegraphics[width=.9\textwidth]{paste-23935852740609.png}
\includegraphics[width=.9\textwidth]{paste-23948737642497.png}

\end{tcolorbox}
%---------------------
\begin{tcolorbox}[colback=white!10!white,colframe=lightgray!75!black,
  savelowerto=\jobname_ex.tex]

\begin{center}
Fassen Sie die verschiedenen Arten von Betriebssystemen zusammen.

\end{center}

\tcblower

\justifying
\includegraphics[width=.9\textwidth]{paste-24163486007297.png}
\includegraphics[width=.9\textwidth]{paste-24146306138113.png}

\end{tcolorbox}
%---------------------
\begin{tcolorbox}[colback=white!10!white,colframe=lightgray!75!black,
  savelowerto=\jobname_ex.tex]

\begin{center}
Was ist ein 
\textbf{Kernel
}und woraus bestehen seine wesentlichen Aufgaben?

\end{center}

\tcblower

\justifying
\textbf{Zusammenfassung - Kernel:
}Software Kern des Betriebssystems, ist selbst ein Programm welches Dienste bereitstellt.
\includegraphics[width=.9\textwidth]{paste-24313809862657.png}

\end{tcolorbox}
%---------------------
\begin{tcolorbox}[colback=white!10!white,colframe=lightgray!75!black,
  savelowerto=\jobname_ex.tex]

\begin{center}
Nennen Sie die Hauptarten von BS-Kernel.

\end{center}

\tcblower

\justifying
\includegraphics[width=.9\textwidth]{paste-24902220382209 (1).png}

\end{tcolorbox}
%---------------------
\begin{tcolorbox}[colback=white!10!white,colframe=lightgray!75!black,
  savelowerto=\jobname_ex.tex]

\begin{center}
Beschreiben Sie die 
\textit{Philosophie, Nachteile
}und die 
\textit{Vorteile
}eines 
\textbf{monolithischen Kernel BS
}.

\end{center}

\tcblower

\justifying
\includegraphics[width=.9\textwidth]{paste-25082609008641 (1).png}

\end{tcolorbox}
%---------------------
\begin{tcolorbox}[colback=white!10!white,colframe=lightgray!75!black,
  savelowerto=\jobname_ex.tex]

\begin{center}
Beschreiben Sie die
\textit{Philosophie, Nachteile
}und die
\textit{Vorteile
}eines
\textbf{Mikrokernel BS
}.

\end{center}

\tcblower

\justifying
\includegraphics[width=.9\textwidth]{paste-25232932864001.png}

\end{tcolorbox}
%---------------------
\begin{tcolorbox}[colback=white!10!white,colframe=lightgray!75!black,
  savelowerto=\jobname_ex.tex]

\begin{center}
Beschreiben Sie die
\textit{Philosophie, Nachteile
}und die
\textit{Vorteile
}eines 
\textbf{hybridenBS Kernels
}.

\end{center}

\tcblower

\justifying
\includegraphics[width=.9\textwidth]{paste-25701084299265 (1).png}

\end{tcolorbox}
%---------------------
\begin{tcolorbox}[colback=white!10!white,colframe=lightgray!75!black,
  savelowerto=\jobname_ex.tex]

\begin{center}
Fassen Sie noch einmal die verschiedenen Klassen von Kernels zusammen und grenzen Sie diese voneinander ab.

\end{center}

\tcblower

\justifying
\includegraphics[width=.9\textwidth]{paste-25756918874113.png}
\includegraphics[width=.9\textwidth]{paste-25769803776001.png}
\includegraphics[width=.9\textwidth]{paste-25786983645185.png}
\includegraphics[width=.9\textwidth]{paste-25799868547073.png}

\end{tcolorbox}
%---------------------
\begin{tcolorbox}[colback=white!10!white,colframe=lightgray!75!black,
  savelowerto=\jobname_ex.tex]

\begin{center}
Nennen Sie die wesentlichen Vorteile von 
\textbf{UEFI
} gegenüber 
\textbf{BIOS
}!<missing IDENTIFIER>

\end{center}

\tcblower

\justifying
\includegraphics[width=.9\textwidth]{betriebssysteme_skript_49.png}

\end{tcolorbox}
%---------------------
\begin{tcolorbox}[colback=white!10!white,colframe=lightgray!75!black,
  savelowerto=\jobname_ex.tex]

\begin{center}
Betriebssysteme::Lernkontrolle

\end{center}

\tcblower

\justifying

\end{tcolorbox}
%---------------------
\begin{tcolorbox}[colback=white!10!white,colframe=lightgray!75!black,
  savelowerto=\jobname_ex.tex]

\begin{center}
Nennen Sie die wesentlichen Dienstleistungen eines Betriebssystems

\end{center}

\tcblower

\justifying
\includegraphics[width=.9\textwidth]{betriebssysteme_skript_49.png}

\end{tcolorbox}
%---------------------
\begin{tcolorbox}[colback=white!10!white,colframe=lightgray!75!black,
  savelowerto=\jobname_ex.tex]

\begin{center}
Welche der Kernel Aufgaben gehört 
\textbf{NICHT
} zu den Aufgaben eines BS-Kerns?

\end{center}

\tcblower

\justifying
\includegraphics[width=.9\textwidth]{betriebssysteme_skript_50.png}

\end{tcolorbox}
%---------------------
\begin{tcolorbox}[colback=white!10!white,colframe=lightgray!75!black,
  savelowerto=\jobname_ex.tex]

\begin{center}
Beschreiben Sie die drei Hauptarten von BS-Kernel und gehen Sie dabei auf die Vor- und Nachteile ein!

\end{center}

\tcblower

\justifying
\includegraphics[width=.9\textwidth]{betriebssysteme_skript_50.png}

\end{tcolorbox}
%---------------------
\begin{tcolorbox}[colback=white!10!white,colframe=lightgray!75!black,
  savelowerto=\jobname_ex.tex]

\begin{center}
Definieren Sie den Begriff des 
\textbf{Prozesses
}. 
Was versteht man in diesem Zusammenhang unter der der 
\textit{Quasiparallelität
}?<missing IDENTIFIER>

\end{center}

\tcblower

\justifying
\includegraphics[width=.9\textwidth]{betriebssysteme_skript_54.png}
\includegraphics[width=.9\textwidth]{betriebssysteme_skript_54.png}

\end{tcolorbox}
%---------------------
\begin{tcolorbox}[colback=white!10!white,colframe=lightgray!75!black,
  savelowerto=\jobname_ex.tex]

\begin{center}
Was versteht man unter dem sog. 
\textbf{Prozessmodell
}?<missing IDENTIFIER>
Erläutern Sie in diesem Zusammenhang grob, was man unter einer 
\textit{Schedulingstrategie
} versteht.

\end{center}

\tcblower

\justifying
\includegraphics[width=.9\textwidth]{betriebssysteme_skript_54.png}
\includegraphics[width=.9\textwidth]{betriebssysteme_skript_54.png}

\end{tcolorbox}
%---------------------
\begin{tcolorbox}[colback=white!10!white,colframe=lightgray!75!black,
  savelowerto=\jobname_ex.tex]

\begin{center}
Beschreiben Sie den Begriff der 
\textbf{Prozesstabelle
}.
Welche Daten werden hierbei festgehalten?

\end{center}

\tcblower

\justifying
\includegraphics[width=.9\textwidth]{betriebssysteme_skript_55.png}

\end{tcolorbox}
%---------------------
\begin{tcolorbox}[colback=white!10!white,colframe=lightgray!75!black,
  savelowerto=\jobname_ex.tex]

\begin{center}
Was versteht man unter einem 
\textit{Prozesskontrollblock
}?<missing IDENTIFIER>

\end{center}

\tcblower

\justifying
\includegraphics[width=.9\textwidth]{betriebssysteme_skript_55.png}
\includegraphics[width=.9\textwidth]{betriebssysteme_skript_56.png}

\end{tcolorbox}
%---------------------
\begin{tcolorbox}[colback=white!10!white,colframe=lightgray!75!black,
  savelowerto=\jobname_ex.tex]

\begin{center}
Erläutern Sie den Begriff der 
\textbf{Prozesserzeugung
}. 
Wann ist so etwas sinnvoll, welche Möglichkeiten der Erzeugung gibt es, welche Befehle werden verwendet?

\end{center}

\tcblower

\justifying
\includegraphics[width=.9\textwidth]{betriebssysteme_skript_57.png}

\end{tcolorbox}
%---------------------
\begin{tcolorbox}[colback=white!10!white,colframe=lightgray!75!black,
  savelowerto=\jobname_ex.tex]

\begin{center}
Wie kann ein Prozess beendet werden?
Welche unterschiedlichen Arten gibt es, erläutern Sie dies am Beispiel vom BS Linux.

\end{center}

\tcblower

\justifying
\includegraphics[width=.9\textwidth]{betriebssysteme_skript_58.png}

\end{tcolorbox}
%---------------------
\begin{tcolorbox}[colback=white!10!white,colframe=lightgray!75!black,
  savelowerto=\jobname_ex.tex]

\begin{center}
Erläutern Sie die unterschiedlichen Prozesshierarchien von Windows und Unix-Maschinen.
Wie können solche Prozesse erzeugt werden und wie sieht deren Beziehung untereinander aus.

\end{center}

\tcblower

\justifying
\includegraphics[width=.9\textwidth]{betriebssysteme_skript_59.png}

\end{tcolorbox}
%---------------------
\begin{tcolorbox}[colback=white!10!white,colframe=lightgray!75!black,
  savelowerto=\jobname_ex.tex]

\begin{center}
Erläutern Sie das grundsätzliche Zustandsmodell der möglichen Prozesszustände anhand einer Skizze.

\end{center}

\tcblower

\justifying
\includegraphics[width=.9\textwidth]{betriebssysteme_skript_60.png}

\end{tcolorbox}
%---------------------
\begin{tcolorbox}[colback=white!10!white,colframe=lightgray!75!black,
  savelowerto=\jobname_ex.tex]

\begin{center}
Erläutern Sie anhand eines einfachen Beispiels die Sinnhaftigkeit von Threads.

\end{center}

\tcblower

\justifying
\includegraphics[width=.9\textwidth]{betriebssysteme_skript_63.png}

\end{tcolorbox}
%---------------------
\begin{tcolorbox}[colback=white!10!white,colframe=lightgray!75!black,
  savelowerto=\jobname_ex.tex]

\begin{center}
Worin genau liegt der Unterschied zwischen 
\textit{Prozessen
} und 
\textit{Threads
}?<missing IDENTIFIER>

\end{center}

\tcblower

\justifying
\includegraphics[width=.9\textwidth]{betriebssysteme_skript_64.png}
\includegraphics[width=.9\textwidth]{betriebssysteme_skript_65.png}

\end{tcolorbox}
%---------------------
\begin{tcolorbox}[colback=white!10!white,colframe=lightgray!75!black,
  savelowerto=\jobname_ex.tex]

\begin{center}
Nennen Sie die beiden Möglichkeiten der Thread-erzeugung unter Linux. 
Worin liegt der Unterschied zur Erzeugung von Prozessen?

\end{center}

\tcblower

\justifying
\includegraphics[width=.9\textwidth]{betriebssysteme_skript_66.png}
\includegraphics[width=.9\textwidth]{betriebssysteme_skript_67.png}
\includegraphics[width=.9\textwidth]{betriebssysteme_skript_68.png}
\includegraphics[width=.9\textwidth]{betriebssysteme_skript_69.png}
\includegraphics[width=.9\textwidth]{betriebssysteme_skript_70.png}

\end{tcolorbox}
%---------------------
\begin{tcolorbox}[colback=white!10!white,colframe=lightgray!75!black,
  savelowerto=\jobname_ex.tex]

\begin{center}
Was versteht man unter einem sog. 
\textbf{Scheduler
}?<missing IDENTIFIER>

\end{center}

\tcblower

\justifying
\includegraphics[width=.9\textwidth]{betriebssysteme_skript_72.png}
Das 
\textit{Kleingedruckte
}: 
Permanent am Arbeiten, muss selber bei der CPU rechnend aktiv werden um anderen Prozessen ihre Zuteilung bereitzustellen.

\end{tcolorbox}
%---------------------
\begin{tcolorbox}[colback=white!10!white,colframe=lightgray!75!black,
  savelowerto=\jobname_ex.tex]

\begin{center}
Auf welchen Systemen ist ein 
\textbf{Scheduler
} von besonderer Bedeutung? Gehen Sie hierbei insbesondere auf die Gründe ein.

\end{center}

\tcblower

\justifying
\includegraphics[width=.9\textwidth]{betriebssysteme_skript_73.png}

\end{tcolorbox}
%---------------------
\begin{tcolorbox}[colback=white!10!white,colframe=lightgray!75!black,
  savelowerto=\jobname_ex.tex]

\begin{center}
Erläutern Sie den Begriff des 
\textbf{Prozessverhaltens
} im Zusammenhang mit dem 
\textit{Scheduler
}.

\end{center}

\tcblower

\justifying
\includegraphics[width=.9\textwidth]{betriebssysteme_skript_74.png}

\end{tcolorbox}
%---------------------
\begin{tcolorbox}[colback=white!10!white,colframe=lightgray!75!black,
  savelowerto=\jobname_ex.tex]

\begin{center}
Welche Möglichkeiten gibt es 
\textit{wann
} das Scheduling beginnen sollte?

\end{center}

\tcblower

\justifying
\includegraphics[width=.9\textwidth]{betriebssysteme_skript_74.png}
\includegraphics[width=.9\textwidth]{betriebssysteme_skript_75.png}

\end{tcolorbox}
%---------------------
\begin{tcolorbox}[colback=white!10!white,colframe=lightgray!75!black,
  savelowerto=\jobname_ex.tex]

\begin{center}
Grenzen Sie die 
\textit{unterbrechbaren
} von den 
\textit{nicht-unterbrechbaren
} Strategien ab.

\end{center}

\tcblower

\justifying
\includegraphics[width=.9\textwidth]{betriebssysteme_skript_75.png}

\end{tcolorbox}
%---------------------
\begin{tcolorbox}[colback=white!10!white,colframe=lightgray!75!black,
  savelowerto=\jobname_ex.tex]

\begin{center}
Nennen Sie die beiden Möglichkeiten der Thread-Erzeugung unter Linux. 
Worin liegt der Unterschied zur Erzeugung von Prozessen?

\end{center}

\tcblower

\justifying
\includegraphics[width=.9\textwidth]{betriebssysteme_skript_66.png}
\includegraphics[width=.9\textwidth]{betriebssysteme_skript_67.png}
\includegraphics[width=.9\textwidth]{betriebssysteme_skript_68.png}
\includegraphics[width=.9\textwidth]{betriebssysteme_skript_69.png}
\includegraphics[width=.9\textwidth]{betriebssysteme_skript_70.png}

\end{tcolorbox}
%---------------------
\begin{tcolorbox}[colback=white!10!white,colframe=lightgray!75!black,
  savelowerto=\jobname_ex.tex]

\begin{center}
Erläutern Sie die möglichen Kriterien für ein gutes 
\textbf{präemtives
} Scheduling. Grenzen Sie dabei noch einmal die beiden wesentlichen 
\textit{Scheduling-Strategien
} voneinander ab.

\end{center}

\tcblower

\justifying
\includegraphics[width=.9\textwidth]{betriebssysteme_skript_77.png}
\includegraphics[width=.9\textwidth]{betriebssysteme_skript_78.png}
\includegraphics[width=.9\textwidth]{betriebssysteme_skript_79.png}

\end{tcolorbox}
%---------------------
\begin{tcolorbox}[colback=white!10!white,colframe=lightgray!75!black,
  savelowerto=\jobname_ex.tex]

\begin{center}
Welche der Kriterien für ein gutes 
\textbf{präemtives
} Scheduling schliessen sich gegenseitig aus? Nennen Sie zwei Beispiele.

\end{center}

\tcblower

\justifying
\includegraphics[width=.9\textwidth]{betriebssysteme_skript_79.png}

\end{tcolorbox}
%---------------------
\begin{tcolorbox}[colback=white!10!white,colframe=lightgray!75!black,
  savelowerto=\jobname_ex.tex]

\begin{center}
Erläutern Sie die Grundidee sowie die Vor- / Nachteile der 
\textbf{FCFS
} Scheduling-Strategie. In welche Kategorie lässt sich diese einordnen? Wofür steht die 
\textit{Abkürzung
}?<missing IDENTIFIER>
Nennen Sie eine Situation aus dem realen Leben die mit dieser Strategie gut verglichen werden kann.

\end{center}

\tcblower

\justifying
\includegraphics[width=.9\textwidth]{betriebssysteme_skript_80.png}

\end{tcolorbox}
%---------------------
\begin{tcolorbox}[colback=white!10!white,colframe=lightgray!75!black,
  savelowerto=\jobname_ex.tex]

\begin{center}
Erläutern Sie die Grundidee sowie die Vor- / Nachteile der 
\textbf{SJF
} Scheduling-Strategie. In welche Kategorie lässt sich diese einordnen? Wofür steht die 
\textit{Abkürzung
}?<missing IDENTIFIER>
Nennen Sie eine Situation aus dem realen Leben die mit dieser Strategie gut verglichen werden kann.

\end{center}

\tcblower

\justifying
\includegraphics[width=.9\textwidth]{betriebssysteme_skript_81.png}

\end{tcolorbox}
%---------------------
\begin{tcolorbox}[colback=white!10!white,colframe=lightgray!75!black,
  savelowerto=\jobname_ex.tex]

\begin{center}
Erläutern Sie die Grundidee sowie die Vor- / Nachteile des 
\textbf{Zeitschreibeverfahrens
}. In welche Kategorie lässt sich diese Strategie einordnen?
Nennen Sie eine Situation aus dem realen Leben die mit dieser Strategie gut verglichen werden kann.
Unter welchem 
\textit{Namen
} ist diese Strategie allgemeinhin noch bekannt?
Nennen Sie ein Beispiel einer 
\textbf{Variation
} dieses Verfahrens.

\end{center}

\tcblower

\justifying
\includegraphics[width=.9\textwidth]{betriebssysteme_skript_82.png}

\end{tcolorbox}
%---------------------
\begin{tcolorbox}[colback=white!10!white,colframe=lightgray!75!black,
  savelowerto=\jobname_ex.tex]

\begin{center}
Was versteht man im Zusammenhang des 
\textbf{Zeitschreibeverfahrens
} unter einem 
\textit{Quantum
}?<missing IDENTIFIER>

\end{center}

\tcblower

\justifying
\includegraphics[width=.9\textwidth]{betriebssysteme_skript_82.png}
\includegraphics[width=.9\textwidth]{betriebssysteme_skript_83.png}
\includegraphics[width=.9\textwidth]{betriebssysteme_skript_84.png}

\end{tcolorbox}
%---------------------
\begin{tcolorbox}[colback=white!10!white,colframe=lightgray!75!black,
  savelowerto=\jobname_ex.tex]

\begin{center}
Erläutern Sie den Begriff sowie die 
\textbf{Grundidee
} des 
\textbf{MQS
}, wofür steht diese 
\textit{Abkürzung
}?<missing IDENTIFIER>

\end{center}

\tcblower

\justifying
\includegraphics[width=.9\textwidth]{betriebssysteme_skript_86.png}

\end{tcolorbox}
%---------------------
\begin{tcolorbox}[colback=white!10!white,colframe=lightgray!75!black,
  savelowerto=\jobname_ex.tex]

\begin{center}
Welche Rolle spielt der Begriff 
\textbf{Feedback
} in der 
\textbf{Multilevel 
\textit{Feedback
} Queue Scheduling
} Strategie?

\end{center}

\tcblower

\justifying
\includegraphics[width=.9\textwidth]{betriebssysteme_skript_86.png}
\includegraphics[width=.9\textwidth]{betriebssysteme_skript_87.png}
\includegraphics[width=.9\textwidth]{betriebssysteme_skript_88.png}
\includegraphics[width=.9\textwidth]{betriebssysteme_skript_89.png}

\end{tcolorbox}
%---------------------
\begin{tcolorbox}[colback=white!10!white,colframe=lightgray!75!black,
  savelowerto=\jobname_ex.tex]

\begin{center}
Erläutern Sie die 
\textbf{MLFQ
}-Strategie anhand von 
\textit{3 Queues
} und dem 
\textit{Zeitscheibeverfahren
}.

\end{center}

\tcblower

\justifying
\includegraphics[width=.9\textwidth]{betriebssysteme_skript_89.png}

\end{tcolorbox}
%---------------------
\begin{tcolorbox}[colback=white!10!white,colframe=lightgray!75!black,
  savelowerto=\jobname_ex.tex]

\begin{center}
Nennen Sie die 
\textit{wichtigsten
} Parameter der 
\textbf{MLFQ
}-Strategie.

\end{center}

\tcblower

\justifying
\includegraphics[width=.9\textwidth]{betriebssysteme_skript_90.png}

\end{tcolorbox}
%---------------------
\begin{tcolorbox}[colback=white!10!white,colframe=lightgray!75!black,
  savelowerto=\jobname_ex.tex]

\begin{center}
Beschreiben Sie die 
\textbf{Ausgangssituation
} sowie die 
\textit{Kernidee
} eines 
\textbf{CFS
}. Wofür steht die 
\textit{Abkürzung?
}
\end{center}

\tcblower

\justifying
\includegraphics[width=.9\textwidth]{betriebssysteme_skript_93.png}

\end{tcolorbox}
%---------------------
\begin{tcolorbox}[colback=white!10!white,colframe=lightgray!75!black,
  savelowerto=\jobname_ex.tex]

\begin{center}
Wie berechnet sich die 
\textit{virtuelle
} Laufzeit eines 
\textbf{CFS
} und wo liegt der Zusammenhang zur 
\textit{Priorität
} eines Prozesses?

\end{center}

\tcblower

\justifying
\includegraphics[width=.9\textwidth]{betriebssysteme_skript_93.png}

\end{tcolorbox}
%---------------------
\begin{tcolorbox}[colback=white!10!white,colframe=lightgray!75!black,
  savelowerto=\jobname_ex.tex]

\begin{center}
Welche 
\textbf{Datenstruktur
} verwendet ein 
\textbf{CFS
} um seine Daten zu speichern?

\end{center}

\tcblower

\justifying
\includegraphics[width=.9\textwidth]{betriebssysteme_skript_94.png}

\end{tcolorbox}
%---------------------
\begin{tcolorbox}[colback=white!10!white,colframe=lightgray!75!black,
  savelowerto=\jobname_ex.tex]

\begin{center}
Erläutern Sie das Konzept einer 
\textbf{Task Group
} im Zusammenhang mit dem 
\textbf{CFS
}.

\end{center}

\tcblower

\justifying
\includegraphics[width=.9\textwidth]{betriebssysteme_skript_95.png}

\end{tcolorbox}
%---------------------
\begin{tcolorbox}[colback=white!10!white,colframe=lightgray!75!black,
  savelowerto=\jobname_ex.tex]

\begin{center}
Geben Sie einen 
\textit{kurzen
} Überblick über den 
\textbf{BFS
}.

\end{center}

\tcblower

\justifying
\includegraphics[width=.9\textwidth]{betriebssysteme_skript_97.png}

\end{tcolorbox}
%---------------------
\begin{tcolorbox}[colback=white!10!white,colframe=lightgray!75!black,
  savelowerto=\jobname_ex.tex]

\begin{center}
Beschreiben Sie grob was man unter dem sog. 
\textbf{Nebenläufigkeitsproblem
} versteht. Erläutern Sie dies anhand eines Beispiels.

\end{center}

\tcblower

\justifying
\includegraphics[width=.9\textwidth]{betriebssysteme_skript_99.png}
\includegraphics[width=.9\textwidth]{betriebssysteme_skript_103.png}
\includegraphics[width=.9\textwidth]{betriebssysteme_skript_104.png}
\includegraphics[width=.9\textwidth]{betriebssysteme_skript_108.png}

\end{tcolorbox}
%---------------------
\begin{tcolorbox}[colback=white!10!white,colframe=lightgray!75!black,
  savelowerto=\jobname_ex.tex]

\begin{center}
Nennen Sie die beiden Möglichkeiten der Thread-Erzeugung und Linux. 
Worin liegt der Unterschied zur Erzeugung von Prozessen?

\end{center}

\tcblower

\justifying
\includegraphics[width=.9\textwidth]{betriebssysteme_skript_66.png}
\includegraphics[width=.9\textwidth]{betriebssysteme_skript_67.png}
\includegraphics[width=.9\textwidth]{betriebssysteme_skript_68.png}
\includegraphics[width=.9\textwidth]{betriebssysteme_skript_69.png}
\includegraphics[width=.9\textwidth]{betriebssysteme_skript_70.png}

\end{tcolorbox}
%---------------------
\begin{tcolorbox}[colback=white!10!white,colframe=lightgray!75!black,
  savelowerto=\jobname_ex.tex]

\begin{center}
Charakterisieren Sie die drei wesentlichen Probleme der 
\textbf{Nebenläufigkeit
}.

\end{center}

\tcblower

\justifying
\includegraphics[width=.9\textwidth]{betriebssysteme_skript_112.png}
\includegraphics[width=.9\textwidth]{betriebssysteme_skript_113.png}
\includegraphics[width=.9\textwidth]{betriebssysteme_skript_114.png}
\includegraphics[width=.9\textwidth]{betriebssysteme_skript_115.png}

\end{tcolorbox}
%---------------------
\begin{tcolorbox}[colback=white!10!white,colframe=lightgray!75!black,
  savelowerto=\jobname_ex.tex]

\begin{center}
Erläutern Sie was man unter einem sog. 
\textit{kritischem Abschnitt
} versteht. Nennen Sie den 
\textit{Fachbegriff
} des Lösungsansatzes und beschreiben Sie die 4 
\textbf{Bedingungen
} für eine 
\textit{gute
} Lösung.

\end{center}

\tcblower

\justifying
\includegraphics[width=.9\textwidth]{betriebssysteme_skript_118.png}

\end{tcolorbox}
%---------------------
\begin{tcolorbox}[colback=white!10!white,colframe=lightgray!75!black,
  savelowerto=\jobname_ex.tex]

\begin{center}
Erläutern Sie wieso es das 
\textbf{abschalten
} von Interrupts zum Steueren eines 
\textit{kritischen Abschnitts
} nicht als optimaler Lösungsansatz angesehen wird. Wo kann dieser Ansatz jedoch sinnvoll sein?

\end{center}

\tcblower

\justifying
\includegraphics[width=.9\textwidth]{betriebssysteme_skript_120.png}

\end{tcolorbox}
%---------------------
\begin{tcolorbox}[colback=white!10!white,colframe=lightgray!75!black,
  savelowerto=\jobname_ex.tex]

\begin{center}
Erläutern Sie die 
\textit{Strategie
} des 
\textit{Sperrens von Variablen
} zum regeln des Zugriffs auf einen kritischen Abschnitt.

\end{center}

\tcblower

\justifying
\includegraphics[width=.9\textwidth]{betriebssysteme_skript_120.png}
\includegraphics[width=.9\textwidth]{betriebssysteme_skript_121.png}
\includegraphics[width=.9\textwidth]{betriebssysteme_skript_122.png}
Das 
\textit{Kleingedruckte
}: Scheduler könnte im falschen Moment unterbrechen.
\includegraphics[width=.9\textwidth]{betriebssysteme_skript_122.png}

\end{tcolorbox}
%---------------------
\begin{tcolorbox}[colback=white!10!white,colframe=lightgray!75!black,
  savelowerto=\jobname_ex.tex]

\begin{center}
Beschreiben Sie die 
\textit{Lösung von Peterson
} zum regeln des Zugriffs auf einen kritischen Abschnitt. Auf wie viele 
\textit{Threads
} kann dieser Ansatz angewendet werden? Beschreiben Sie neben den Vorteilen auch die Nachteile dieses Ansatzes.

\end{center}

\tcblower

\justifying
\includegraphics[width=.9\textwidth]{betriebssysteme_skript_126.png}
\includegraphics[width=.9\textwidth]{betriebssysteme_skript_127.png}
\includegraphics[width=.9\textwidth]{betriebssysteme_skript_128.png}
\includegraphics[width=.9\textwidth]{betriebssysteme_skript_129.png}
\includegraphics[width=.9\textwidth]{betriebssysteme_skript_130.png}

\end{tcolorbox}
%---------------------
\begin{tcolorbox}[colback=white!10!white,colframe=lightgray!75!black,
  savelowerto=\jobname_ex.tex]

\begin{center}
Richtig oder Falsch: Die Abfolge von 
\textit{Programmabläufen
} unterschiedlicher Prozesse kann immer durch deren Reihenfolge im Source-Code 
\textit{reproduzierbar
} festgelegt werden?

\end{center}

\tcblower

\justifying
Falsch: Durch die Zuordnung der Prozesse zur CPU und Prozesse mit höherer Priorität sind zeitliche Programmabläufe selten repduzierbar.

\end{tcolorbox}
%---------------------
\begin{tcolorbox}[colback=white!10!white,colframe=lightgray!75!black,
  savelowerto=\jobname_ex.tex]

\begin{center}
Beschreiben Sie die Aufgaben eines 
\textbf{Prozesskontrollblocks
} (PCB) und geben sie an 
\textit{wo
} dieser abgespeichert wird.

\end{center}

\tcblower

\justifying
PCB ist eine Datenstruktur, die genutzt wird, um assoziierte dAten für einen Prozess zu speichern.
PCBs werden in der Prozesstabelle des BS gespeichert.

\end{tcolorbox}
%---------------------
\begin{tcolorbox}[colback=white!10!white,colframe=lightgray!75!black,
  savelowerto=\jobname_ex.tex]

\begin{center}
Nennen und beschreiben Sie die drei Zustände eines Prozesses gemäss dem Standardschema für das Zustandsmodell.

\end{center}

\tcblower

\justifying
- Bereit
- Rechnend
- Blockiert

\end{tcolorbox}
%---------------------
\begin{tcolorbox}[colback=white!10!white,colframe=lightgray!75!black,
  savelowerto=\jobname_ex.tex]

\begin{center}
Richtig oder Falsch: Ein Thread hat einen eigenen Stack und eingene CPU Register, Threads eines Prozesses teilen sich aber den gleichen Speicher innerhalb dieses Prozesses.

\end{center}

\tcblower

\justifying
Richtig!

\end{tcolorbox}
%---------------------
\begin{tcolorbox}[colback=white!10!white,colframe=lightgray!75!black,
  savelowerto=\jobname_ex.tex]

\begin{center}
Was wird mit dem UNIX-Befehl 
\textit{fork
} erzeugt?

\end{center}

\tcblower

\justifying
Eine exakte Kopie des Elternprozesses.

\end{tcolorbox}
%---------------------
\begin{tcolorbox}[colback=white!10!white,colframe=lightgray!75!black,
  savelowerto=\jobname_ex.tex]

\begin{center}
Was versteht man im Zusammenhang mit der Verarbeitung eines 
\textbf{kritischen Abschnitts
} unter 
\textit{TSL-Operationen?
}
\end{center}

\tcblower

\justifying
\includegraphics[width=.9\textwidth]{betriebssysteme_skript_132.png}
\includegraphics[width=.9\textwidth]{betriebssysteme_skript_133.png}
\includegraphics[width=.9\textwidth]{betriebssysteme_skript_134.png}

\end{tcolorbox}
%---------------------
\begin{tcolorbox}[colback=white!10!white,colframe=lightgray!75!black,
  savelowerto=\jobname_ex.tex]

\begin{center}
Erläutern Sie die Lösungsmöglichkeit für den 
\textit{wechselseitigen Ausschluss
} mittels einer 
\textbf{Semaphore
}. Welche 
\textit{Operationen
} stellt dieses Verfahren zur Verfügung?

\end{center}

\tcblower

\justifying
\includegraphics[width=.9\textwidth]{betriebssysteme_skript_136.png}
\includegraphics[width=.9\textwidth]{betriebssysteme_skript_137.png}
\includegraphics[width=.9\textwidth]{betriebssysteme_skript_138.png}
\includegraphics[width=.9\textwidth]{betriebssysteme_skript_140.png}

\end{tcolorbox}
%---------------------
\begin{tcolorbox}[colback=white!10!white,colframe=lightgray!75!black,
  savelowerto=\jobname_ex.tex]

\begin{center}
Erläutern Sie die 
\textit{verschiedenen
} Möglichkeiten zur Realisierung von 
\textbf{atomaren
} Operationen.
Beschreiben Sie in diesem Zusammenhang noch kurz den 
\textit{Unterschied
} zwischen der sog. 
\textit{Quasi-Paralleltiät
} sowie der 
\textit{echten Paralleltität
}.

\end{center}

\tcblower

\justifying
\includegraphics[width=.9\textwidth]{betriebssysteme_skript_141.png}

\end{tcolorbox}
%---------------------
\begin{tcolorbox}[colback=white!10!white,colframe=lightgray!75!black,
  savelowerto=\jobname_ex.tex]

\begin{center}
Beschreiben Sie den zeitlichen Ablauf des 
\textit{wechselseiten Ausschlusses
} unter verwendung einer 
\textbf{Semaphore
} anhand eines Beispiels.

\end{center}

\tcblower

\justifying
\includegraphics[width=.9\textwidth]{betriebssysteme_skript_136.png}
\includegraphics[width=.9\textwidth]{betriebssysteme_skript_142.png}
\includegraphics[width=.9\textwidth]{betriebssysteme_skript_145.png}
\includegraphics[width=.9\textwidth]{betriebssysteme_skript_146.png}
\includegraphics[width=.9\textwidth]{betriebssysteme_skript_147.png}

\end{tcolorbox}
%---------------------
\begin{tcolorbox}[colback=white!10!white,colframe=lightgray!75!black,
  savelowerto=\jobname_ex.tex]

\begin{center}
Beschreiben Sie das Konzept des sog. 
\textbf{Monitors
}. Wofür ist es geeignet und welches bekannte Verfahren soll es ersetzen?
Gehen Sie auch auf die Nachteile von Monitoren ein.

\end{center}

\tcblower

\justifying
\includegraphics[width=.9\textwidth]{betriebssysteme_skript_149.png}
\includegraphics[width=.9\textwidth]{betriebssysteme_skript_150.png}

\end{tcolorbox}

\end{document}


